\documentclass[a4paper,12pt]{article}
%%%%%%%%%%%%%%%%%%%%%%%%%%%%%%%%%%%%%%%%%%%%%%%%%%%%%%%%%%%%%%%%%%%%%%%%%%%%%%%%%%%%%%%%%%%%%%%%%%%%%%%%%%%%%%%%%%%%%%%%%%%%%%%%%%%%%%%%%%%%%%%%%%%%%%%%%%%%%%%%%%%%%%%%%%%%%%%%%%%%%%%%%%%%%%%%%%%%%%%%%%%%%%%%%%%%%%%%%%%%%%%%%%%%%%%%%%%%%%%%%%%%%%%%%%%%
\usepackage{eurosym}
\usepackage{vmargin}
\usepackage{amsmath}
\usepackage{graphics}
\usepackage{epsfig}
\usepackage{enumerate}
\usepackage{multicol}
\usepackage{subfigure}
\usepackage{fancyhdr}
\usepackage{listings}
\usepackage{framed}
\usepackage{graphicx}
\usepackage{amsmath}
\usepackage{chngpage}
%\usepackage{bigints}

\usepackage{vmargin}
% left top textwidth textheight headheight
% headsep footheight footskip
\setmargins{2.0cm}{2.5cm}{16 cm}{22cm}{0.5cm}{0cm}{1cm}{1cm}
\renewcommand{\baselinestretch}{1.3}

\setcounter{MaxMatrixCols}{10}

\begin{document}

```R
Q. 3)
Mortality of a group of lives follows Gompertz’s law. 

The value of the parameters is known as c = 1.128 & B = 7.29 x 10 -6 .
1
Use the approximation q x = 1 − e −μ(x+ 2 ) , calculate an approximate of the curtate value
for age 55 or e55. Assuming 100 is the maximum age.

Solution 3:


```


```R
i)
B=0.00000729
C=1.128
gmu<-function(x){
Mu<-B*C^x
Mu
}
qx<-function(x){
1-exp(-gmu(x+1/2))
}



```


```R
x<-55
Page 5 of 11IAI

ex<-0
npx<-1
for(i in 1:(100-x)){
px=1-qx(x+i-1)
npx=npx*px
ex<-npx+ex}
ex
> ex
 21.71408
```
