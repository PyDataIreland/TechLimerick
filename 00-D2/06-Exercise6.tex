\documentclass[a4paper,12pt]{article}
%%%%%%%%%%%%%%%%%%%%%%%%%%%%%%%%%%%%%%%%%%%%%%%%%%%%%%%%%%%%%%%%%%%%%%%%%%%%%%%%%%%%%%%%%%%%%%%%%%%%%%%%%%%%%%%%%%%%%%%%%%%%%%%%%%%%%%%%%%%%%%%%%%%%%%%%%%%%%%%%%%%%%%%%%%%%%%%%%%%%%%%%%%%%%%%%%%%%%%%%%%%%%%%%%%%%%%%%%%%%%%%%%%%%%%%%%%%%%%%%%%%%%%%%%%%%
\usepackage{eurosym}
\usepackage{vmargin}
\usepackage{amsmath}
\usepackage{graphics}
\usepackage{epsfig}
\usepackage{enumerate}
\usepackage{multicol}
\usepackage{subfigure}
\usepackage{fancyhdr}
\usepackage{listings}
\usepackage{framed}
\usepackage{graphicx}
\usepackage{amsmath}
\usepackage{chngpage}
%\usepackage{bigints}

\usepackage{vmargin}
% left top textwidth textheight headheight
% headsep footheight footskip
\setmargins{2.0cm}{2.5cm}{16 cm}{22cm}{0.5cm}{0cm}{1cm}{1cm}
\renewcommand{\baselinestretch}{1.3}

\setcounter{MaxMatrixCols}{10}

\begin{document}
Suppose the transition probabilities are a function of the age of a person. The transition
probability of a person aged x moving:

* from Healthy to Sick state is 0.007x
* from Healthy to Death state is 0.001x
* from Sick to Death state is 0.002(100-x)
* from Sick to Healthy is 0.006(100-x).

Assuming 100 is the maximum age. Calculate the probability of:

i) Healthy person aged 30 will be in Sick state after 4 years.

ii) Sick person aged 25 will be in Death state after 7 years.




```R

H2S<-function(x){ 0.007*x}
H2D<-function(x){ 0.001*x}
S2H<-function(x){ 0.006*(100-x)}
S2D<-function(x){ 0.002*(100-x)}



```


```R

transmat<-function(x){
  M<-matrix(0,nrow=3,ncol=3)
  M[1,1] <- 1 - H2S(x) - H2D(x)
  M[1,2] <- H2S(x)
  M[1,3] <- H2D(x)
  M[2,1] <- S2H(x)
  M[2,2] <- 1 - S2H(x) - S2D(x)
  M[2,3] <- S2D(x) 
  M[3,1] <- 0
  M[3,2] <- 0
  M[3,3] <- 1
  
}


```


```R
M
```


    Error in eval(expr, envir, enclos): object 'M' not found
    Traceback:




```R

n=30
B<-c(1,0,0)
for (i in 1:4){
B=B%*%transmat(n+i-1)}
B
```


    Error in B %*% transmat(n + i - 1): non-conformable arguments
    Traceback:




```R


> B
[,1]
[,2]
[,3]
[1,] 0.5451076 0.2608138 0.1940787
Hence the probability of Healthy person aged 30 will be in Sick state after 4 years is 0.2608138.

[9]
ii)
n=25
C<-c(0,1,0)
for (i in 1:7){
C=C%*%transmat(n+i-1)}
C
> C
[,1]
[,2]
[,3]
[1,] 0.3946025 0.1720589 0.4333386
Hence the probability of sick person aged 25 will be in Death state after 7 years is 0.4333386.

```
