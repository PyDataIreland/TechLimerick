\documentclass[a4paper,12pt]{article}
%%%%%%%%%%%%%%%%%%%%%%%%%%%%%%%%%%%%%%%%%%%%%%%%%%%%%%%%%%%%%%%%%%%%%%%%%%%%%%%%%%%%%%%%%%%%%%%%%%%%%%%%%%%%%%%%%%%%%%%%%%%%%%%%%%%%%%%%%%%%%%%%%%%%%%%%%%%%%%%%%%%%%%%%%%%%%%%%%%%%%%%%%%%%%%%%%%%%%%%%%%%%%%%%%%%%%%%%%%%%%%%%%%%%%%%%%%%%%%%%%%%%%%%%%%%%
\usepackage{eurosym}
\usepackage{vmargin}
\usepackage{amsmath}
\usepackage{graphics}
\usepackage{epsfig}
\usepackage{enumerate}
\usepackage{multicol}
\usepackage{subfigure}
\usepackage{fancyhdr}
\usepackage{listings}
\usepackage{framed}
\usepackage{graphicx}
\usepackage{amsmath}
\usepackage{chngpage}
%\usepackage{bigints}

\usepackage{vmargin}
% left top textwidth textheight headheight
% headsep footheight footskip
\setmargins{2.0cm}{2.5cm}{16 cm}{22cm}{0.5cm}{0cm}{1cm}{1cm}
\renewcommand{\baselinestretch}{1.3}

\setcounter{MaxMatrixCols}{10}

\begin{document}

```R
Q. 5)
(5)
(7)
[12]
Signals are mathematically represented by a function that takes into account the magnitude
and frequency.
x t = A cos(2πωt + φ) + w t , where ω is equal to 1/50 and w t is the white noise with mean
zero and SD of 5. A and φ need to be estimated.
In this case the parameters appear in a nonlinear way, so we use a trigonometric identity
and write
A cos(2πωt + φ) = A cos(φ) cos(2πωt) - A sin(φ) sin(2πωt),
= β 1 cos(2πωt) + β 2 sin(2πωt),
= β 1 *z1 + β 2 *z2
If we can estimate β 1 and β 2 , then A and φ can be estimated.
Therefore, x t = β 1 cos(2πωt) + β 2 sin(2πωt) + w t
i)
Simulate and plot a series x t , using A=2, φ = 0.6π and SD of white noise as 5 (for t= 1
to 1000).
ii) Generate two series z1 and z2 (for t= 1 to 1000). (3)
iii) Fit a Linear model to between x t , z1 and z2 and estimate parameters β1, β2 , A and φ. (7)
iv) Plot x, fitted line.
```


```R



Solution 5:
i)
set.seed(90210) # so you can reproduce these results
x = 2*cos(2*pi*1:1000/50 + .6*pi) + rnorm(1000,0,5)
plot.ts(x)

ii)
z1 = cos(2*pi*1:1000/50)
z2 = sin(2*pi*1:1000/50)

```


```R



iii)
summary(fit <- lm(x~0+z1+z2)) # zero to exclude the intercept
> summary(fit <- lm(x~0+z1+z2)) # zero to exclude the intercept
Call:
lm(formula = x ~ 0 + z1 + z2)
Residuals:
Min
1Q
-16.0671 -3.9817
Median
-0.4643
3Q
3.0976
Max
15.7080
Coefficients:
Estimate Std. Error t value Pr(>|t|)
z1 -0.6784
0.2278 -2.978 0.00297 **
z2 -2.2071
0.2278 -9.690 < 2e-16 ***
---
Signif. codes: 0 ‘***’ 0.001 ‘**’ 0.01 ‘*’ 0.05 ‘.’ 0.1 ‘ ’ 1
Residual standard error: 5.093 on 998 degrees of freedom
Page 7 of 11IAI
Multiple R-squared: 0.09336, Adjusted R-squared: 0.09154
F-statistic: 51.38 on 2 and 998 DF, p-value: < 2.2e-16
```


```R


[4]
beta1 = -0.6784
beta2 = -2.2071
phi = atan(-beta2/beta1)
phi
A = beta1/cos(phi)
A
> phi
 -1.27259
> A
 -2.309008
iv)
plot.ts(x, col=8, ylab=expression(hat(x)))
lines(fitted(fit), col=2)

```
