\documentclass[a4paper,12pt]{article}
%%%%%%%%%%%%%%%%%%%%%%%%%%%%%%%%%%%%%%%%%%%%%%%%%%%%%%%%%%%%%%%%%%%%%%%%%%%%%%%%%%%%%%%%%%%%%%%%%%%%%%%%%%%%%%%%%%%%%%%%%%%%%%%%%%%%%%%%%%%%%%%%%%%%%%%%%%%%%%%%%%%%%%%%%%%%%%%%%%%%%%%%%%%%%%%%%%%%%%%%%%%%%%%%%%%%%%%%%%%%%%%%%%%%%%%%%%%%%%%%%%%%%%%%%%%%
\usepackage{eurosym}
\usepackage{vmargin}
\usepackage{amsmath}
\usepackage{graphics}
\usepackage{epsfig}
\usepackage{enumerate}
\usepackage{multicol}
\usepackage{subfigure}
\usepackage{fancyhdr}
\usepackage{listings}
\usepackage{framed}
\usepackage{graphicx}
\usepackage{amsmath}
\usepackage{chngpage}
%\usepackage{bigints}

\usepackage{vmargin}
% left top textwidth textheight headheight
% headsep footheight footskip
\setmargins{2.0cm}{2.5cm}{16 cm}{22cm}{0.5cm}{0cm}{1cm}{1cm}
\renewcommand{\baselinestretch}{1.3}

\setcounter{MaxMatrixCols}{10}

\begin{document}

```R
library(magrittr)

options(repr.plot.width=8, repr.plot.height=5)
```

For the graphs, the value of x varies from 0 to 10.

## Part (a)

Draw the graph of hazard function for the gamma distribution with parameters $\alpha  = 0.75$ and $\lambda  = 0.5$.


The hazard function $h(x)$ for a distribution is defined as the ratio between its probability density function and its survival function.

Survival functions are most often used in reliability and related fields. The survival function is the probability that the variate takes a value greater than x.

$$S(x)=Pr[X&gt;x]=1−F(x)$$

#### Probaility Density Function


```R
pdf <-function(x) {dgamma(x,0.75,0.3)}

plot.function(pdf,0,10)
```


![png](output_5_0.png)


#### Survival Function


```R
sf <-function(x){ 1-pgamma(x,0.75,0.3) }

plot.function(sf,0,10)
title("Survival Function")

```


![png](output_7_0.png)


#### Hazard Function


```R
h1<-function(x) {dgamma(x,0.75,0.3)/(1-pgamma(x,0.75,0.3))}

plot.function(h1,0,10)
title("Hazard Function 1")


```


![png](output_9_0.png)


## Part (b)

Draw the graph of hazard function for the gamma distribution with parameters $\alpha  = 1$ and $\lambda  = 0.3$. 


```R
h2<-function(x) {dgamma(x,1,0.3)/(1-pgamma(x,1,0.3))}

plot.function(h2,0,10)
```


![png](output_11_0.png)


## Part (c)

Draw the graph of hazard function for the gamma distribution with parameters $\alpha  = 1.5$ and $\lambda  = 0.3$. 


```R
h3<-function(x) {dgamma(x,1.5,0.3)/(1-pgamma(x,1.5,0.3))}

plot.function(h3,0,10)
```


![png](output_13_0.png)


## Part (d)

Comment on the thickness of the tails of the distribution of ***Part (b)*** with the tails of ***Part (a)*** and
***Part (c)***.


* If $\alpha  &lt; 1$, it is a decreasing function of x and thus indicating a heavier tail than the exponential
distribution. 


* If $\alpha  &gt; 1$, it is a increasing function of x and thus indicating a lighter tail than the exponential
distribution. 


* If $\alpha  = 1$ then the function is relatively constant in most part of the projection.



```R

```
