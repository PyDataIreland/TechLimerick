\documentclass[a4paper,12pt]{article}
%%%%%%%%%%%%%%%%%%%%%%%%%%%%%%%%%%%%%%%%%%%%%%%%%%%%%%%%%%%%%%%%%%%%%%%%%%%%%%%%%%%%%%%%%%%%%%%%%%%%%%%%%%%%%%%%%%%%%%%%%%%%%%%%%%%%%%%%%%%%%%%%%%%%%%%%%%%%%%%%%%%%%%%%%%%%%%%%%%%%%%%%%%%%%%%%%%%%%%%%%%%%%%%%%%%%%%%%%%%%%%%%%%%%%%%%%%%%%%%%%%%%%%%%%%%%
\usepackage{eurosym}
\usepackage{vmargin}
\usepackage{amsmath}
\usepackage{graphics}
\usepackage{epsfig}
\usepackage{enumerate}
\usepackage{multicol}
\usepackage{subfigure}
\usepackage{fancyhdr}
\usepackage{listings}
\usepackage{framed}
\usepackage{graphicx}
\usepackage{amsmath}
\usepackage{chngpage}
%\usepackage{bigints}

\usepackage{vmargin}
% left top textwidth textheight headheight
% headsep footheight footskip
\setmargins{2.0cm}{2.5cm}{16 cm}{22cm}{0.5cm}{0cm}{1cm}{1cm}
\renewcommand{\baselinestretch}{1.3}

\setcounter{MaxMatrixCols}{10}

\begin{document}

### Question 1
Data is available for the movement of taxis in Dublin. 

The city is divided into three zones “North”, “South” and “West”. (*The “East” is Dublin Bay*)

The movement of a taxi from one zone to another will depend only on its current position. 

The following probabilities have been determined for taxi movements:


* Of all taxis in the North zone, 30% will remain in North and 30% will move to South, with the remaining 40% moving to West.

* In the South zone, taxis have a 40% chance of moving to North, 40% chance of staying in South and 20% chance of moving to West.

* Of all the drivers in the West zone, 50% will move to North and 30% to South with the remaining 20% staying in West.

The movement of taxis in Dublin will be modelled in R using a Markov Chain.




#### Driver Movement
A 3 - dimensional discrete Markov Chain defined by the following states:
North, South, West

The transition matrix (by rows) is defined as follows:
\begin{array}{cccc}
&amp; North &amp; South &amp; West\\
North &amp; 0.3 &amp; 0.3 &amp; 0.4\\
South &amp; 0.4 &amp; 0.4 &amp; 0.2\\
West &amp; 0.5 &amp; 0.3 &amp; 0.2\\
\end{array}

## Exercise 1

Create a vector with the state space of the Markov Chain, using R code. You should print this to the screen and paste into your answer.


```R

DriverZone <- c("North", "South", "West")
DriverZone

```


<ol class="list-inline">
	<li>'North'</li>
	<li>'South'</li>
	<li>'West'</li>
</ol>



## Exercise 2

Construct a transition matrix of the zone movement probabilities. 
You should print this to the screen and paste into your answer.


```R
ZoneTransition <- matrix(c(0.3, 0.3, 0.4, 0.4, 0.4, 0.2,0.5, 0.3, 0.2), 
                         nrow = 3, 
                         byrow = TRUE)

ZoneTransition
```


<table>
<caption>A matrix: 3 x 3 of type dbl</caption>
<tbody>
	<tr><td>0.3</td><td>0.3</td><td>0.4</td></tr>
	<tr><td>0.4</td><td>0.4</td><td>0.2</td></tr>
	<tr><td>0.5</td><td>0.3</td><td>0.2</td></tr>
</tbody>
</table>




```R
ZoneTransition <- matrix(c(0.3, 0.3, 0.4, 0.4, 0.4, 0.2,0.5, 0.3, 0.2), 
                         nrow = 3, 
                         byrow = TRUE, 
                         dimname = list(DriverZone, DriverZone))

ZoneTransition
```


<table>
<caption>A matrix: 3 x 3 of type dbl</caption>
<thead>
	<tr><th></th><th>North</th><th>South</th><th>West</th></tr>
</thead>
<tbody>
	<tr><th>North</th><td>0.3</td><td>0.3</td><td>0.4</td></tr>
	<tr><th>South</th><td>0.4</td><td>0.4</td><td>0.2</td></tr>
	<tr><th>West</th><td>0.5</td><td>0.3</td><td>0.2</td></tr>
</tbody>
</table>



## Exercise 3

Load the R package for Markov Chains: {``markovchain``}. 

N.B. ``R`` is case sensitive .



```R
sessionInfo()
```


    R version 3.5.3 (2019-03-11)
    Platform: x86_64-pc-linux-gnu (64-bit)
    Running under: Ubuntu 16.04.6 LTS
    
    Matrix products: default
    BLAS: /opt/microsoft/ropen/3.5.3/lib64/R/lib/libRblas.so
    LAPACK: /opt/microsoft/ropen/3.5.3/lib64/R/lib/libRlapack.so
    
    locale:
     [1] LC_CTYPE=en_US.UTF-8       LC_NUMERIC=C              
     [3] LC_TIME=en_US.UTF-8        LC_COLLATE=en_US.UTF-8    
     [5] LC_MONETARY=en_US.UTF-8    LC_MESSAGES=en_US.UTF-8   
     [7] LC_PAPER=en_US.UTF-8       LC_NAME=C                 
     [9] LC_ADDRESS=C               LC_TELEPHONE=C            
    [11] LC_MEASUREMENT=en_US.UTF-8 LC_IDENTIFICATION=C       
    
    attached base packages:
    [1] stats     graphics  grDevices utils     datasets  methods   base     
    
    loaded via a namespace (and not attached):
     [1] Rcpp_1.0.2      digest_0.6.22   zeallot_0.1.0   crayon_1.3.4   
     [5] IRdisplay_0.7.0 repr_1.0.1      backports_1.1.5 jsonlite_1.6   
     [9] evaluate_0.14   pillar_1.4.2    rlang_0.4.1     uuid_0.1-2     
    [13] vctrs_0.2.0     IRkernel_1.0.2  tools_3.5.3     compiler_3.5.3 
    [17] pkgconfig_2.0.3 base64enc_0.1-3 htmltools_0.4.0 pbdZMQ_0.3-3   
    [21] tibble_2.1.3   


### Installing the {markovchain} R package

<pre><code>
install.packages("markovchain") 

library(markovchain)
</code></pre>



Requires R Version 3.6 

## Exercise 4

Create a Markov Chain object with state space equal to your vector in ***Exercise 1*** and transition matrix from ***Exercise 2***. 

``
MCZone &lt;- new("markovchain", states = DriverZone, 
              byrow = TRUE, 
              transitionMatrix = ZoneTransition, 
              name = "DriverMovement")``

``
MCZone
``

## Exercise  5. 

Calculate the probability that a driver currently in the North zone will be in the
North zone after:

1. two trips

2. three trips.



```R
DriverZone <- c("North", "South", "West")
ZoneTransition <- matrix(c(0.3, 0.3, 0.4, 
                           0.4, 0.4, 0.2,
                           0.5, 0.3, 0.2), 
                         nrow = 3, 
                         byrow = T, 
                         dimname = list(DriverZone, DriverZone))


```

WRONG


```R
# Wrong
ZoneTransition^2
```

CORRECT


```R
ZoneTransition %*% ZoneTransition
```


```R
ZoneTransition %*% ZoneTransition %*% ZoneTransition
```


```R

MCZone^2

MCZone^3

```

* The required probability in 2 trips is 41% or 0.41

* The required probability in 3 trips is 38.5% or 0.385


## Exercise 6

Determine the stationary state of the Markov Chain.




```R
steadyStates(MCZone)

```

### Linear Algebra

* $\pi$ is the $1\times n$ steady-state equilibrium row vector
* $P$ is the $n \times n$ transition matrix
* $I$ is the $n \times n$ Identity matrix


$$\begin{eqnarray*}
        \pi P &amp;=&amp; \pi   \qquad (  \pi \mbox{ is unchanged by  P }.) \\
              &amp;=&amp; \pi I \\
              &amp; &amp; \\
\pi ( P − I ) &amp;=&amp; \boldsymbol{0}  \\
\end{eqnarray*}$$

Use transposition to re-arrange as a standard system of linear equations in form $Ax=b$

$$ ( P − I )^{T} \times \pi^{T} \;=\; \boldsymbol{0} $$

* $( P − I )^{T}$ is an $n \times n$ matrix
* $\pi^{T}$ is an $n \times 1$ column vector
* $\boldsymbol{0}$ is an $n \times 1$ column vector of 0s


```R
#diag(3)
```


```R
t(ZoneTransition - diag(3))
```


<table>
<caption>A matrix: 3 x 3 of type dbl</caption>
<tbody>
	<tr><td>-0.7</td><td> 0.4</td><td> 0.5</td></tr>
	<tr><td> 0.3</td><td>-0.6</td><td> 0.3</td></tr>
	<tr><td> 0.4</td><td> 0.2</td><td>-0.8</td></tr>
</tbody>
</table>




```R
det( t(ZoneTransition - diag(3)) )
```


1.71291552370738e-17


Solve a system of linear equations: ``solve(A,b)``


```R
A <- t(ZoneTransition) - diag(3)
b <- c(0,0,0)

cbind( A, b )
```


<table>
<caption>A matrix: 3 x 4 of type dbl</caption>
<thead>
	<tr><th></th><th></th><th></th><th>b</th></tr>
</thead>
<tbody>
	<tr><td>-0.7</td><td> 0.4</td><td> 0.5</td><td>0</td></tr>
	<tr><td> 0.3</td><td>-0.6</td><td> 0.3</td><td>0</td></tr>
	<tr><td> 0.4</td><td> 0.2</td><td>-0.8</td><td>0</td></tr>
</tbody>
</table>




```R
det(A)
```


1.71291552370738e-17



```R
#solve( A, b )
```

We also can use the $\sum \pi = 1$

$$ \pi_1 + \pi_2 + \ldots +  \pi_n = 1$$


```R
A <- rbind( t(ZoneTransition - diag(3)), c(1,1,1) )
A
```


<table>
<caption>A matrix: 4 x 3 of type dbl</caption>
<tbody>
	<tr><td>-0.7</td><td> 0.4</td><td> 0.5</td></tr>
	<tr><td> 0.3</td><td>-0.6</td><td> 0.3</td></tr>
	<tr><td> 0.4</td><td> 0.2</td><td>-0.8</td></tr>
	<tr><td> 1.0</td><td> 1.0</td><td> 1.0</td></tr>
</tbody>
</table>




```R
b <- c(0,0,0,1)
cbind(A,b)
```


<table>
<caption>A matrix: 4 x 4 of type dbl</caption>
<thead>
	<tr><th></th><th></th><th></th><th>b</th></tr>
</thead>
<tbody>
	<tr><td>-0.7</td><td> 0.4</td><td> 0.5</td><td>0</td></tr>
	<tr><td> 0.3</td><td>-0.6</td><td> 0.3</td><td>0</td></tr>
	<tr><td> 0.4</td><td> 0.2</td><td>-0.8</td><td>0</td></tr>
	<tr><td> 1.0</td><td> 1.0</td><td> 1.0</td><td>1</td></tr>
</tbody>
</table>




```R
qr.solve( A, b )

```


<ol class="list-inline">
	<li>0.388888888888889</li>
	<li>0.333333333333333</li>
	<li>0.277777777777778</li>
</ol>




```R

```
