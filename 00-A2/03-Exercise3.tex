\documentclass[a4paper,12pt]{article}
%%%%%%%%%%%%%%%%%%%%%%%%%%%%%%%%%%%%%%%%%%%%%%%%%%%%%%%%%%%%%%%%%%%%%%%%%%%%%%%%%%%%%%%%%%%%%%%%%%%%%%%%%%%%%%%%%%%%%%%%%%%%%%%%%%%%%%%%%%%%%%%%%%%%%%%%%%%%%%%%%%%%%%%%%%%%%%%%%%%%%%%%%%%%%%%%%%%%%%%%%%%%%%%%%%%%%%%%%%%%%%%%%%%%%%%%%%%%%%%%%%%%%%%%%%%%
\usepackage{eurosym}
\usepackage{vmargin}
\usepackage{amsmath}
\usepackage{graphics}
\usepackage{epsfig}
\usepackage{enumerate}
\usepackage{multicol}
\usepackage{subfigure}
\usepackage{fancyhdr}
\usepackage{listings}
\usepackage{framed}
\usepackage{graphicx}
\usepackage{amsmath}
\usepackage{chngpage}
%\usepackage{bigints}

\usepackage{vmargin}
% left top textwidth textheight headheight
% headsep footheight footskip
\setmargins{2.0cm}{2.5cm}{16 cm}{22cm}{0.5cm}{0cm}{1cm}{1cm}
\renewcommand{\baselinestretch}{1.3}

\setcounter{MaxMatrixCols}{10}

\begin{document}

\begin{framed} \begin{verbatim}
library(magrittr)
\end{verbatim}\end{framed}


You have been asked to analyse the percentage changes in quarterly personal
consumption expenditure and personal disposable income in the US from 1970–2010.

This information is contained in a time series called “consumption” in R’s `{fpp}` library.
The data can be loaded into R using the command:




\begin{framed} \begin{verbatim}
#install.packages("fpp")
#library(fpp)
\end{verbatim}\end{framed}


\begin{framed} \begin{verbatim}
usconsumption %>% head()
\end{verbatim}\end{framed}


<table>
<caption>A Time Series: 6 x 2</caption>
<thead>
	<tr><th></th><th>consumption</th><th>income</th></tr>
</thead>
<tbody>
	<tr><th>1970 Q1</th><td> 0.6122769</td><td> 0.496540</td></tr>
	<tr><th>1970 Q2</th><td> 0.4549298</td><td> 1.736460</td></tr>
	<tr><th>1970 Q3</th><td> 0.8746730</td><td> 1.344881</td></tr>
	<tr><th>1970 Q4</th><td>-0.2725144</td><td>-0.328146</td></tr>
	<tr><th>1971 Q1</th><td> 1.8921870</td><td> 1.965432</td></tr>
	<tr><th>1971 Q2</th><td> 0.9133782</td><td> 1.490757</td></tr>
</tbody>
</table>




\begin{framed} \begin{verbatim}
consumption <- usconsumption[,1]
\end{verbatim}\end{framed}

%%%%%%%%%%%%%%%%%%%%%%5
\newpage 
\subsection*{Exercise 1}

Plot this time series giving appropriate labels for each axis and paste the R
code and the chart into your answer.




\newpage 
* (ii) (a) Plot the Autocorrelation function (ACF) and Partial Autocorrelation function (PACF), giving appropriate labels for each axis and paste the
R code and the charts into your answer.

(b) Comment on your plots in part (ii)(a), making reference to the stationarity.



Fit the most appropriate ARIMA model based on the results in part (ii)(a),
stating the equation of the model and justifying your choice.



%%%%%%%%%%%%%%%%%%%%%%5
\newpage 
The dataset usconsumption also includes the quarterly personal disposable income from 1970 to 2010. 
The data can be loaded into R using the command:



\begin{framed} \begin{verbatim}
income <- usconsumption[,2]
\end{verbatim}\end{framed}


Compare the performance of the ARIMA model you have chosen in part (iii) with a linear regression model of consumption on income, by computing the root mean square error (RMSE) for the fitted values of
each model. Paste your coding and R output into your answer.
(b)
Explain why the RMSE is not the ideal measure to compare these models, including a recommendation of a more appropriate measure.


An analyst has suggested that neither model in part (iv) is a good fit to the data and has asked you to propose an alternative model.

%%%%%%%%%%%%%%%%%%%%%%5
\newpage 
\subsection*{Part 5}
(v)
(a)
(b)
Suggest a suitable alternative model to fit to the data. Fit the model you have suggested in part (v)(a) to the data, stating the
equation used.
(c)
Compare the results of this model to the models fitted in part (iv).

\end{document}
