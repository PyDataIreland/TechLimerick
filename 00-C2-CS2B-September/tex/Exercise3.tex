\documentclass[a4paper,12pt]{article}
%%%%%%%%%%%%%%%%%%%%%%%%%%%%%%%%%%%%%%%%%%%%%%%%%%%%%%%%%%%%%%%%%%%%%%%%%%%%%%%%%%%%%%%%%%%%%%%%%%%%%%%%%%%%%%%%%%%%%%%%%%%%%%%%%%%%%%%%%%%%%%%%%%%%%%%%%%%%%%%%%%%%%%%%%%%%%%%%%%%%%%%%%%%%%%%%%%%%%%%%%%%%%%%%%%%%%%%%%%%%%%%%%%%%%%%%%%%%%%%%%%%%%%%%%%%%
\usepackage{eurosym}
\usepackage{vmargin}
\usepackage{amsmath}
\usepackage{graphics}
\usepackage{epsfig}
\usepackage{enumerate}
\usepackage{multicol}
\usepackage{subfigure}
\usepackage{fancyhdr}
\usepackage{listings}
\usepackage{framed}
\usepackage{graphicx}
\usepackage{amsmath}
\usepackage{chngpage}
%\usepackage{bigints}

\usepackage{vmargin}
% left top textwidth textheight headheight
% headsep footheight footskip
\setmargins{2.0cm}{2.5cm}{16 cm}{22cm}{0.5cm}{0cm}{1cm}{1cm}
\renewcommand{\baselinestretch}{1.3}

\setcounter{MaxMatrixCols}{10}

\begin{document}



library(magrittr)

In an insurance company’s portfolio, individual claim sizes, in \$, follow an exponential distribution with rate parameter $\lambda = 0.0001$.


In R console run the following code: ``set.seed(123)``

and then use R to simulate 10,000 claims and plot a histogram of the simulated data. 


\begin{verbatim}
set.seed(123)

claims <- rexp(10000, 0.0001)

hist(claims, main = "Histogram of 10,000 Simulated Claims Values from the Exponential Distribution")

\end{verbatim}


## Exercise 2

#### Part (a)

Calculate the mean and the variance of the simulated claims in ***Exercise 1***.


mean(claims)

var(claims)


library(magrittr)  # pipe operator %>%

mean(claims) %>% round(3)

var(claims)   %>% round(3)

library(scales)    # Comma



mean(claims) %>% comma()

var(claims)  %>% comma()


#### Part B

Compare your answers to part (ii)(a) with the theoretical values expected from the exponential distribution with parameter 0.0001.


\begin{itemize}
    \item The theoretical value of the mean should be 1/0.0001 = 10,000.

    \item The theoretical value of the variance should be (1/0.0001^2 ) = 100,000,000 

    \item The simulated values are very close to the theoretical values. 

    \item The difference is due to sampling error 
\end{itemize}


%%%%%%%%%%%%%%%%%%%%%%%%%%%%%%%%%%%%%%%%%%%
\newpage
\subsection*{Exercise 3}

The insurer decides to take out an individual excess of loss reinsurance arrangement with a retention level of $20,000.


Calculate the mean and the variance of the claims paid, under this arrangement, by
(a) the insurer
(b) the reinsurer.

The insurer wishes to determine an appropriate retention limit and has asked an analyst to investigate the effect of different retention limits on the mean and variance
of claims.


(iv)
Calculate the mean and variance of the claims paid by:
(a)
(b)
the insurer
the reinsurer
under each of the following six retention levels:
£5,000, £10,000, £20,000, £30,000, £40,000 and £50,000.


(iii)(a)
For the insurer:
\begin{framed} \begin{verbatim}
InsClaims3 <- pmin(claims, 20000)
mean(InsClaims3)
 8671.328
var(InsClaims3)
 43964610
```

(b)
For the reinsurer:
ReClaims3 <- pmax(0, claims - 20000)
mean(ReClaims3)
 1366.485
var(ReClaims3)
 25047814

OR: (alternative accepted answer based on conditional distribution).
ReClaims3_alt <- claims[claims>20000]-20000
mean(ReClaims3_alt)
 10099.67
var(ReClaims3_alt)
 96978902

%%%%%%%%%%%%%%%%%%%%%%%%%%%%%%%%%%%%%%

(iv)(a)
InsClaims1
InsClaims2
InsClaims4
InsClaims5
InsClaims6
<-
<-
<-
<-
<-
pmin(claims,
pmin(claims,
pmin(claims,
pmin(claims,
pmin(claims,
5000)
10000)
30000)
40000)
50000)
MeanIns <- c(mean(InsClaims1), mean(InsClaims2),
mean(InsClaims3), mean(InsClaims4), mean(InsClaims5),
mean(InsClaims6))
MeanIns
 3940.658 6339.937 8671.328 9538.658 9869.412
9971.611
VarIns <- c(var(InsClaims1), var(InsClaims2),
var(InsClaims3), var(InsClaims4), var(InsClaims5),
var(InsClaims6))
VarIns
 2533775 12858446 43964610 70204248 86376566
93370367

(b)
ReClaims1 <- pmax(0, claims - 5000)
ReClaims2 <- pmax(0, claims - 10000)
ReClaims4 <- pmax(0, claims - 30000)
ReClaims5 <- pmax(0, claims - 40000)
ReClaims6 <- pmax(0, claims - 50000)
MeanRe <- c(mean(ReClaims1), mean(ReClaims2),
mean(ReClaims3), mean(ReClaims4), mean(ReClaims5),
mean(ReClaims6))
MeanRe
 6097.15469 3697.87535 1366.48492
168.40057
66.20174
499.15454
VarRe <- c(var(ReClaims1), var(ReClaims2),
var(ReClaims3), var(ReClaims4), var(ReClaims5),
var(ReClaims6))
VarRe
 84523429 60046376 25047814
1305645
9343406
3450843

OR: (alternative accepted answer based on conditional distribution).
ReClaims1_alt <- claims[claims>5000] -5000
ReClaims2_alt <- claims[claims>10000]-10000


ReClaims4_alt <- claims[claims>30000]-30000
ReClaims5_alt <- claims[claims>40000]-40000
ReClaims6_alt <- claims[claims>50000]-50000
MeanRe_alt <- c(mean(ReClaims1_alt),mean(ReClaims2_alt),
mean(ReClaims3_alt), mean(ReClaims4_alt),
mean(ReClaims5_alt), mean(ReClaims6_alt))
MeanRe_alt
 10019.975 10040.389 10099.667
11033.623
9562.348
9568.214
VarRe_alt <- c(var(ReClaims1_alt), var(ReClaims2_alt),
var(ReClaims3_alt), var(ReClaims4_alt),
var(ReClaims5_alt), var(ReClaims6_alt))
VarRe_alt
 99600486 99365957 96978902
98212292


\end{verbatim}
\end{framed}

%%%%%%%%%%%%%%%%%%%%%%%%%%%%%%%
\newpage 


[8]
(v) Plot your results from part (iv) on four separate charts to show how the mean
and variance of the claims paid by the insurer and reinsurer vary with the
retention level.

(v)
92486321 106717925

\begin{framed}
\begin{verbatim}
x <- c(5000, 10000, 20000, 30000, 40000, 50000)
plot(x, MeanIns, xlab = "Retention Limit", ylab = "Mean
amount paid by Insurer", main = "Mean Claim Amount for
Insurer by Retention Limit", type = "l")
plot(x, MeanRe, xlab = "Retention Limit", ylab = "Mean
amount paid by Reinsurer",main = "Mean Claim Amount for
Reinsurer by Retention Limit", type = "l")
plot(x, VarIns, xlab = "Retention Limit", ylab =
"Variance of amount paid by Insurer", main = "Variance
of Claim Amount for Insurer by Retention Limit", type =
"l")
plot(x, VarRe, xlab = "Retention Limit", ylab =
"Variance of amount paid by Reinsurer", main = "Variance
of Claim Amount for Reinsurer by Retention Limit", type
= "l")

\end{verbatim}
\end{framed}


%%%%%%%%%%%%%%%%%%%%%%%
\newpage 

OR: Alternative accepted graph for the reinsurer based on conditional distribution.
\begin{framed}
\begin{verbatim}
x <- c(5000, 10000, 20000, 30000, 40000, 50000)
plot(x, MeanRe_alt, xlab = "Retention Limit", ylab = "Mean
amount paid by Reinsurer",main = "Mean Claim Amount for
Reinsurer by Retention Limit", type = "l")
plot(x, VarRe_alt, xlab = "Retention Limit", ylab = "Variance
of amount paid by Reinsurer", main = "Variance of Claim Amount
for Reinsurer by Retention Limit", type = "l")    
\end{verbatim}
\end{framed}







(vi)
 Comment, with reference to the total variance, on your results in part (v). [6]
%----------------------------------------%
–4

vi)
The mean claim amount paid by the insurer increases in size with the retention limit
but at a decreasing rate.

It tends towards the unrestricted mean as the retention limit increases 
The mean claim amount paid by the reinsurer is equal to the total mean claim
minus the mean amount paid by the insurer 
Hence the mean claim amount paid by the reinsurer decreases in size with the
retention limit also at a decreasing rate. 
The trends in the variance of the claims for the insurer and the reinsurer
are not “mirror images”. 
Among the six retention limits investigated, the sum of the variance of the reinsurer
and the insurer reaches a minimum at a retention limit of around £20,000


(
TotalVar <- VarIns + VarRe
TotalVar
 87057204 72904822 69012424 79547654 89827408
94676012

This suggests that this retention limit might be the most appropriate in practice. 
[Max 6]
OR: Alternative relevant solutions based on conditional distribution of the
reinsurer is accepted. In this case,
Among the six retention limits investigated, the sum of the variance of the reinsurer
and the insurer is at a minimum at a retention limit of around £5,000


TotalVar_alt <- VarIns + VarRe_alt


TotalVar_alt
102134261 112224403 140943512 162690569 193094490
191582658

This suggests that this retention limit might be the most appropriate in practice. 
%%%%%%%%%%%%%%%%%%%%%%%%%%%%%%%%%%%%%%%%%%%%5
\newpage 

Parts (i) to (iv) were very well answered.

Although most candidates scored highly in part (ii), only a few candidates received full marks. Most candidates lost marks for not explaining why the mean and variance of the simulated claims were not the same as the theoretical values.
Answers to part (v) were mixed. Graphs had to be clearly labelled to score full marks.

Part (vi) was answered poorly. Many candidates did not comment on the total variance, despite being asked to do so in the question. Appropriate alternative comments received
credit here. To be appropriate, the comments had to relate to the results in part (v).

In parts (iii) to (vi) full credit was awarded to candidates who calculated the mean and variance of the Reinsurer’s claims based on the conditional distribution i.e. only claims
over the retention limit.


\end{document}