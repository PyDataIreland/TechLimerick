\documentclass[a4paper,12pt]{article}
%%%%%%%%%%%%%%%%%%%%%%%%%%%%%%%%%%%%%%%%%%%%%%%%%%%%%%%%%%%%%%%%%%%%%%%%%%%%%%%%%%%%%%%%%%%%%%%%%%%%%%%%%%%%%%%%%%%%%%%%%%%%%%%%%%%%%%%%%%%%%%%%%%%%%%%%%%%%%%%%%%%%%%%%%%%%%%%%%%%%%%%%%%%%%%%%%%%%%%%%%%%%%%%%%%%%%%%%%%%%%%%%%%%%%%%%%%%%%%%%%%%%%%%%%%%%
\usepackage{eurosym}
\usepackage{vmargin}
\usepackage{amsmath}
\usepackage{graphics}
\usepackage{epsfig}
\usepackage{enumerate}
\usepackage{multicol}
\usepackage{subfigure}
\usepackage{fancyhdr}
\usepackage{listings}
\usepackage{framed}
\usepackage{graphicx}
\usepackage{amsmath}
\usepackage{chngpage}
%\usepackage{bigints}

\usepackage{vmargin}
% left top textwidth textheight headheight
% headsep footheight footskip
\setmargins{2.0cm}{2.5cm}{16 cm}{22cm}{0.5cm}{0cm}{1cm}{1cm}
\renewcommand{\baselinestretch}{1.3}

\setcounter{MaxMatrixCols}{10}

\begin{document}


\section{Introduction}

Data
Before answering this question, the data file must be generated in R using the following code:

\begin{framed}
\begin{verbatim}
set.seed(12345); 
y = arima.sim( list(order = c(1,1,0), ar = 0.7), n = 300)

set.seed(12345); 
y = arima.sim( list(order = c(1,1,0), ar = 0.7), n = 300)
\end{verbatim}
\end{framed}

Plot the time series giving appropriate labels for each axis and paste the chart into your answer.

\begin{framed}
\begin{verbatim}
ts.plot( y, 
        
        xlab = "Time", 
  
        ylab = "Simulated Values from an ARIMA(1,1,0) Time Series")

plot( y, 
  
      xlab = "Time", 
  
      ylab = "Simulated Values from an ARIMA(1,1,0) Time Series")
​

x <- 1:301
​
plot(x, y, 
     
     type="l", 
     
     xlab = "Time", 
  
     ylab = "Simulated Values from an ARIMA(1,1,0) Time Series")
\end{verbatim}
\end{framed}

\newpage 

(b) Comment on the general features of your chart.

%%%%%%%%%%%%%%%%%%%%%%%%%%%%%%%%%%%%%%
\newpage 
\subsection*{Exercise 2}
(a) Determine the best least squares linear fit, adding it to your chart.

(b) Explain whether this least squares linear trend can be removed such that a zero mean stationary distribution is appropriate for the residuals.

x = 1:301
leastsquaresfit = lm(y~x)
leastsquaresfit$coefficients
(Intercept)
35.3816076916281
x
0.251593150412161
(ii) (a)
​
Intercept: 16.4405647
Slope: -0.2436929 [11⁄2]
abline(leastsquaresfit)
OR:
lines(leastsquaresfit$fitted.values) [11⁄2]
Error in parse(text = x, srcfile = src): <text>:4:22: unexpected input
3: Intercept: 16.4405647
4: Slope: -0.2436929 [11<e2>
                        ^
Traceback:


​
​
(b)
plot(leastsquaresfit$res, xlab = "Time", ylab="Residuals")
​
 
It is clear that the residuals are not stationary as they are negative in the first third
followed by positive residuals in the middle part and then negative in the last part.
​
OR:
acf(leastsquaresfit$res)
The residuals are not stationary as the ACF of the residuals seems to generate a unit root as
the ACF values are very slowly decaying.
​
Error in parse(text = x, srcfile = src): <text>:7:4: unexpected symbol

\newpage 
\subsection*{Exercise 3}
Plot the sample Autocorrelation function (ACF) and sample Partial Autocorrelation function (PACF) of the original data, giving appropriate labels for each axis and paste the charts into your answer.

\begin{verbatim}
acf(y, xlab = "Lag", ylab = "ACF of Simulated Values
from an ARIMA(1,1,0) Time Series", main = "ACF")
pacf(y, xlab = "Lag", ylab = "Partial ACF of Simulated
Values from an ARIMA(1,1,0) Time Series", main = "PACF")   
\end{verbatim}


(iv)
(v)
(a)
(b) Comment, by visually inspecting these plots, on the possible modelling
strategy which could be adopted.
​
(a) Perform an appropriate transformation to the data such that a stationary
model is possible, pasting any relevant charts into your answer.
​
(b) Comment on your answer to part (iv)(a).
​
[5]
(b)


There is clearly not a constant mean / any mean-reverting feature in the data ... 
...so stationarity does not hold.
​
There is perhaps an initial upwards trend...
​
...but overall there seems to be a downwards trend.
​
[7, Max. 4]
​
​
 
(iii)(b)
Despite the PACF showing no significance past lag 2 which could indicate
stationarity...
​
... clearly the ACF is not behaving as a stationary ARMA process should.
​
There is a slow decay in the ACF values suggesting a unit root process
​
So we need to perform differencing.

\newpage 
\subsection*{Exercise 4}


\begin{framed}
\begin{verbatim}
library(forecast)
tsdisplay(diff(y))  
\end{verbatim}
\end{framed}

\begin{framed}
\begin{verbatim}
layout(matrix(c(1,1,2,3), 2, 2, byrow = TRUE))
ts.plot(diff(y), main = "diff(y)")
points(diff(y),cex=0.4)
acf(diff(y))
pacf(diff(y))
\end{verbatim}
\end{framed}


​
​
​
These plots indicate that the differenced data could be stationary and both ACF and
PACF seem to decay fast towards zero.
​
​
 
The plots do not indicate any seasonality.

%%%%%%%%%%%%%%%%%%%%%%%%%%%%%%%%%%%%%%%%%%%%%%55
(v)(a)
The chosen model for the transformed data is ARIMA(1,0,0) since the differenced
data looks stationary, and PACF is close to zero from lag 2.
​

\begin{framed}
\begin{verbatim}
fit10=arima(diff(y),order = c(1,0,0));fit10
\end{verbatim}
\end{framed}

Call:
arima(x = diff(y), order = c(1,0,0))
Coefficients:
ar1 intercept
0.7140
-0.2324
s.e. 0.0402
0.1951
sigma^2 estimated as 0.9493:
aic = 842.46
log likelihood = -418.23,
​
The proposed model parameters for the transformed data are:
α = 0.714, intercept = -0.2324


\newpage 
(a) Propose an appropriate model for the transformed data.
​
(b) Justify the choice of model in part (v)(a) by performing an appropriate
diagnostics procedure and comparisons with alternative models.


%%%%%%%%%%%%%%%%%%%%%%%%%%%%%%%%5

(v)(b)
By inspecting the ACF and PACF plots of differenced data, alternative models can be
considered by changing the values of p and q. In the following we fit 6 models:
​

\begin{framed}
\begin{verbatim}
fit11=arima(diff(y),order= c(1,0,1));fit11$aic
# 844.452
fit10=arima(diff(y),order= c(1,0,0));fit10$aic
# 842.4563
fit01=arima(diff(y),order= c(0,0,1));fit01$aic
# 911.0564
fit21=arima(diff(y),order= c(2,0,1));fit21$aic
# 845.8664
fit12=arima(diff(y),order= c(1,0,2));fit12$aic
# 846.3837
fit22=arima(diff(y),order= c(2,0,2));fit22$aic
# 847.6294
\end{verbatim}
\end{framed}

\begin{itemize}
    \item This confirms that the ARIMA(1,0,0) model is a good fit compared with
these alternatives.

\item To check we can use a diagnostic testing procedure of:
tsdiag(fit10)
\end{itemize}

​
 
...The ACF of residuals together with the corresponding Ljung-Box test output
...(i.e. high p-values observed suggesting good fit - residuals close to white noise)
...suggest that this is a correct model.
[21⁄2]
​
[Total 40]
Part (i) was very well answered. Appropriate alternative comments received credit in part
(i)(b). To be appropriate, the comments had to relate to the general features of the chart
produced in part (i)(a) and not to any other charts.
Part(ii) was less satisfactory. Whilst many candidates were able to plot the least squares
line in part (ii)(a), few commented satisfactorily in part (ii)(b) with many candidates
suggesting that the least squares linear trend could be removed such that the residuals
were stationary.

%%%%%%%%%%%%%%%%%%%%%%%%%%%%%%%%%%%%%%%%%%%%%%%%%%%%%%%%
\newpage 
 
 
 \begin{itemize}
     \item Part (iii) was very well answered. Appropriate alternative comments received credit in part (iii)(b). To be appropriate, the comments had to relate to the ACF and PACF plots
produced in part (iii)(a) and not to any other charts.
​
\item Answers to part (iv) were mixed. Partial credit was awarded to candidates who differenced the time series more than once and compared the variance of each differenced data set. Only partial credit was awarded to candidates who stated that the differenced data
“was stationary” rather than “could be stationary”.
​
\item Part (v) was poorly answered. Many candidates did not fit the correct model in part (v)(a) with most not stating the model parameters for their proposed model. 
\item The proposed model
had to be compared to at least two alternative models to score the full credit in the first part of (v)(b). Few candidates correctly interpreted the diagnositics procedures in the
second part of (v)(b).
 \end{itemize}

​

\end{document}