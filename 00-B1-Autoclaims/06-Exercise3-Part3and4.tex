\documentclass[a4paper,12pt]{article}
%%%%%%%%%%%%%%%%%%%%%%%%%%%%%%%%%%%%%%%%%%%%%%%%%%%%%%%%%%%%%%%%%%%%%%%%%%%%%%%%%%%%%%%%%%%%%%%%%%%%%%%%%%%%%%%%%%%%%%%%%%%%%%%%%%%%%%%%%%%%%%%%%%%%%%%%%%%%%%%%%%%%%%%%%%%%%%%%%%%%%%%%%%%%%%%%%%%%%%%%%%%%%%%%%%%%%%%%%%%%%%%%%%%%%%%%%%%%%%%%%%%%%%%%%%%%
\usepackage{eurosym}
\usepackage{vmargin}
\usepackage{amsmath}
\usepackage{graphics}
\usepackage{epsfig}
\usepackage{enumerate}
\usepackage{multicol}
\usepackage{subfigure}
\usepackage{fancyhdr}
\usepackage{listings}
\usepackage{framed}
\usepackage{graphicx}
\usepackage{amsmath}
\usepackage{chngpage}
%\usepackage{bigints}

\usepackage{vmargin}
% left top textwidth textheight headheight
% headsep footheight footskip
\setmargins{2.0cm}{2.5cm}{16 cm}{22cm}{0.5cm}{0cm}{1cm}{1cm}
\renewcommand{\baselinestretch}{1.3}

\setcounter{MaxMatrixCols}{10}

\begin{document}
\large



\begin{framed}\begin{verbatim}
library(readr)
AutoClaims <- read_csv("AutoClaims.csv")
\end{verbatim} \end{framed}

Your actuarial friend has suggested you to use natural logarithm of “PAID” claims instead of the actual “PAID” Claim amount because the log e (PAID) is more closer to
normal distribution than “PAID” Claims. Verify the statement made by your friend by comparing the Skewness and Excess Kurtosis of both the PAID claims as well as
log e (PAID). Write appropriate custom functions to compute both of them. (7)

Repeat the model in (i) above by considering the suggestion in (iii). Identify and comment on the key differences between both the models. (6)
Your Manager has suggested that the model can be improved by adding interaction effects between STATE and CLASS, STATE and GENDER, CLASS and GENDER
as additional variables to the set of independent variables taken in (i). Evaluate the worthiness of this suggestion.

%%%%%%%%%%%%%%%%%%%%%%%%%%%%%%%%%%%%%%%%%
iii)



\begin{framed}\begin{verbatim}
#Reason for better model
#Checking for the normality of Auto Claims Paid vs. Logairthm of Auto Claims Paid
#Writing Functions for Skewness and Kurtosis
skew<-function(x)mean((x-mean(x))^3)/sd(x)^3
kurt<-function(x)(mean((x-mean(x))^4)/sd(x)^4)-3
skew(AutoClaims$PAID)
\end{verbatim} 
\end{framed}

\begin{framed}
\begin{verbatim}
##  2.619422
kurt(AutoClaims$PAID)
\end{verbatim} 
\end{framed}

\begin{framed}\begin{verbatim}
##  9.20876
skew(log(AutoClaims$PAID))
\end{verbatim} \end{framed}

\begin{framed}\begin{verbatim}
##  0.4528057
kurt(log(AutoClaims$PAID))
\end{verbatim} \end{framed}




Skewness and Kurtosis of Log (Claims) are more close to Zero compared to those of actual
claims paid, thus indicating the possibility of using linear regression with this dependent
variable


iv)
# Using Natural Logarithm of the claims paid
model2<-lm(log(PAID)~.,data = AutoClaims)
summary(model2)
[3]
Page 9 of 13IAI
CS1B-0619
anova(model2)
Analysis of Variance Table
Response: log(PAID)
Df Sum Sq Mean Sq F value Pr(>F)
STATE
10 246.26 24.626 107.6969 < 2.2e-16 ***
CLASS
3 690.09 230.031 1006.0090 < 2.2e-16 ***
GENDER
1 10.88 10.882 47.5908 7.927e-12 ***
AGE
1 0.12 0.123 0.5384 0.4632
Residuals 1401 320.35 0.229
---
Signif. codes: 0 ‘***’ 0.001 ‘**’ 0.01 ‘*’ 0.05 ‘.’ 0.1 ‘ ’ 1
##
## Call:
## lm(formula = log(PAID) ~ ., data = AutoClaims)
##
## Residuals:
## Min
1Q Median
3Q Max
## -0.96098 -0.34264 -0.05047 0.36828 1.08237
##
## Coefficients:
##
Estimate Std. Error t value Pr(>|t|)
## (Intercept) 9.896308 0.168777 58.635 < 2e-16 ***
## STATESTATE 02 -0.154804 0.079889 -1.938 0.0529 .
## STATESTATE 03 0.110585 0.092342 1.198 0.2313
## STATESTATE 04 0.049554 0.085147 0.582 0.5607
## STATESTATE 06 0.116190 0.091239 1.273 0.2031
## STATESTATE 07 0.142721 0.101543 1.406 0.1601
## STATESTATE 10 0.098014 0.107391 0.913 0.3616
## STATESTATE 12 0.027982 0.103105 0.271 0.7861
## STATESTATE 14 0.090316 0.102231 0.883 0.3771
## STATESTATE 15 -0.645918 0.075628 -8.541 < 2e-16 ***
## STATESTATE 17 0.004611 0.085455 0.054 0.9570
## CLASSC11 -1.203098 0.052225 -23.037 < 2e-16 ***
## CLASSC6
-1.988743 0.049829 -39.911 < 2e-16 ***
## CLASSF6
-0.034909 0.062787 -0.556 0.5783
## GENDERM
-0.180923 0.026136 -6.922 6.75e-12 ***
## AGE
0.002145 0.002923 0.734 0.4632
## ---
## Signif. codes: 0 '***' 0.001 '**' 0.01 '*' 0.05 '.' 0.1 ' ' 1
##
## Residual standard error: 0.4782 on 1401 degrees of freedom
## Multiple R-squared: 0.7473, Adjusted R-squared: 0.7446
## F-statistic: 276.2 on 15 and 1401 DF, p-value: < 2.2e-16
Key Differences
1. R-Squared and Adjusted R-Squared improved
fit compared to the initial model
and hence the model is a better
[1.5]
Page 10 of 13IAI
CS1B-0619
2. While a the significance level of a few factor coefficients when compared with the
base categories changed, the overall significant variables did not change which can
be inferred from the ANOVA table
[1.5]
[6]
v)
# Using Interaction effects in the model
model3<-lm(PAID~.+STATE:CLASS+STATE:GENDER+CLASS:GENDER,data = AutoClaims)
summary(model3)
##
## Call:
## lm(formula = PAID ~ . + STATE:CLASS + STATE:GENDER + CLASS:GENDER,
## data = AutoClaims)
##
## Residuals:
## Min
1Q Median
3Q Max
## -13110.3 -1475.9 -377.5 1250.5 20442.8
##
## Coefficients: (3 not defined because of singularities)
##
Estimate Std. Error t value Pr(>|t|)
## (Intercept)
24373.09 3236.06 7.532 9.08e-14 ***
## STATESTATE 02
-7868.94 3269.51 -2.407 0.016227 *
## STATESTATE 03
-3695.87 3422.76 -1.080 0.280427
## STATESTATE 04
6883.00 3545.51 1.941 0.052425 .
## STATESTATE 06
5428.58 3262.57 1.664 0.096363 .
## STATESTATE 07
-979.80 1184.59 -0.827 0.408314
## STATESTATE 10
7340.08 3546.35 2.070 0.038664 *
## STATESTATE 12
1048.26 3382.21 0.310 0.756659
## STATESTATE 14
-2796.37 3538.40 -0.790 0.429494
## STATESTATE 15
-14038.72 3164.04 -4.437 9.86e-06 ***
## STATESTATE 17
-4266.43 3640.98 -1.172 0.241490
## CLASSC11
-15321.38 3534.31 -4.335 1.56e-05 ***
## CLASSC6
-20741.82 3262.53 -6.358 2.79e-10 ***
## CLASSF6
4075.85 3386.90 1.203 0.229026
## GENDERM
-5508.80 1305.95 -4.218 2.62e-05 ***
## AGE
23.06 19.35 1.192 0.233581
## STATESTATE 02:CLASSC11 4114.13 3666.38 1.122 0.262008
## STATESTATE 03:CLASSC11 2022.51 3802.06 0.532 0.594847
## STATESTATE 04:CLASSC11 -7719.04 3904.17 -1.977 0.048229 *
## STATESTATE 06:CLASSC11 -6209.18 3743.46 -1.659 0.097412 .
## STATESTATE 07:CLASSC11 -1049.15 1827.01 -0.574 0.565899
## STATESTATE 10:CLASSC11 -6795.03 3956.71 -1.717 0.086144 .
## STATESTATE 12:CLASSC11 -3756.14 3939.41 -0.953 0.340517
## STATESTATE 14:CLASSC11 1668.66 3927.62 0.425 0.671012
## STATESTATE 15:CLASSC11 7776.13 3581.01 2.171 0.030066 *
## STATESTATE 17:CLASSC11 2227.20 4011.07 0.555 0.578806
## STATESTATE 02:CLASSC6 5677.35 3416.32 1.662 0.096777 .
## STATESTATE 03:CLASSC6 2122.55 3605.64 0.589 0.556177
## STATESTATE 04:CLASSC6 -8231.17 3692.27 -2.229 0.025957 *
## STATESTATE 06:CLASSC6 -6151.60 3417.36 -1.800 0.072066 .
## STATESTATE 07:CLASSC6
NA
NA NA
NA
## STATESTATE 10:CLASSC6 -8117.07 3687.43 -2.201 0.027884 *
[5]
Page 11 of 13IAI
CS1B-0619
## STATESTATE 12:CLASSC6 -2090.95 3561.06 -0.587 0.557187
## STATESTATE 14:CLASSC6 1711.83 3617.52 0.473 0.636143
## STATESTATE 15:CLASSC6 10891.63 3315.76 3.285 0.001046 **
## STATESTATE 17:CLASSC6 2376.31 3776.77 0.629 0.529330
## STATESTATE 02:CLASSF6 -1213.53 3542.37 -0.343 0.731971
## STATESTATE 03:CLASSF6 -2272.17 3835.79 -0.592 0.553708
## STATESTATE 04:CLASSF6 -11888.76 3838.95 -3.097 0.001996 **
## STATESTATE 06:CLASSF6 -17556.04 4687.92 -3.745 0.000188 ***
## STATESTATE 07:CLASSF6
NA
NA NA
NA
## STATESTATE 10:CLASSF6 1619.22 3953.37 0.410 0.682178
## STATESTATE 12:CLASSF6
NA
NA NA
NA
## STATESTATE 14:CLASSF6 -2570.84 3769.07 -0.682 0.495299
## STATESTATE 15:CLASSF6 -3790.79 3441.11 -1.102 0.270822
## STATESTATE 17:CLASSF6 362.96 3914.01 0.093 0.926130
## STATESTATE 02:GENDERM 3404.19 1199.24 2.839 0.004598 **
## STATESTATE 03:GENDERM 3061.44 1347.32 2.272 0.023227 *
## STATESTATE 04:GENDERM 1514.18 1279.13 1.184 0.236716
## STATESTATE 06:GENDERM 2052.42 1337.41 1.535 0.125108
## STATESTATE 07:GENDERM 3094.31 1450.30 2.134 0.033057 *
## STATESTATE 10:GENDERM -73.99 1540.53 -0.048 0.961699
## STATESTATE 12:GENDERM 2476.44 1474.81 1.679 0.093351 .
## STATESTATE 14:GENDERM 2684.73 1624.58 1.653 0.098650 .
## STATESTATE 15:GENDERM 3280.15 1148.55 2.856 0.004357 **
## STATESTATE 17:GENDERM 3033.26 1261.58 2.404 0.016335 *
## CLASSC11:GENDERM
1171.02 730.27 1.604 0.109047
## CLASSC6 :GENDERM
2372.86 692.55 3.426 0.000630 ***
## CLASSF6 :GENDERM
-1786.40 905.11 -1.974 0.048620 *
## ---
## Signif. codes: 0 '***' 0.001 '**' 0.01 '*' 0.05 '.' 0.1 ' ' 1
##
## Residual standard error: 3111 on 1361 degrees of freedom
## Multiple R-squared: 0.8116, Adjusted R-squared: 0.804
## F-statistic: 106.6 on 55 and 1361 DF, p-value: < 2.2e-16
Interpretation
anova(model3)
Analysis of Variance Table
[5]
Response: PAID
Df Sum Sq Mean Sq F value Pr(>F)
STATE
10 9.8027e+09 9.8027e+08 101.2664 < 2.2e-16 ***
CLASS
3 3.7869e+10 1.2623e+10 1304.0318 < 2.2e-16 ***
GENDER
1 4.7271e+08 4.7271e+08 48.8334 4.350e-12 ***
AGE
1 6.6423e+06 6.6423e+06 0.6862 0.407612
STATE:CLASS 27 7.8659e+09 2.9133e+08 30.0960 < 2.2e-16 ***
STATE:GENDER 10 2.7570e+08 2.7570e+07 2.8482 0.001634 **
CLASS:GENDER 3 4.6128e+08 1.5376e+08 15.8841 3.733e-10 ***
Residuals 1361 1.3175e+10 9.6801e+06
---
Signif. codes: 0 ‘***’ 0.001 ‘**’ 0.01 ‘*’ 0.05 ‘.’ 0.1 ‘ ’ 1
1. R-Squared and Adjusted R-Squared increased to above 80% and hence the model is a
better fit compared to the earlier models
[2]
Page 12 of 13IAI
CS1B-0619
2. Interaction effect between a few classes and states emerged out to be very
significant (State 6 and Class F6 came out to be significantly negative). Though State
15 came out to be significantly negative when main effects alone were considered,
the interaction effects compensated that negative significantly when interacted with
class C6 and Class 11 whereas the interaction coefficient is not significant between
State 15 and Class F6 indicating that the claims paid is significantly lesser when the
state is 15 and class is F6 compared to other rating classes. Digging deeper into the
relationships is possible with the interaction effect. Similarly main effect of Gender is
significantly negative compared to females but that is offset to some extent for some
states (2,3,7,15) and for some rating classes (C6) whereas it is further negative in
case of F6. So the differences can be magnified by considering the interaction
effects, improving the predictability of the model

3.
ANOVA table for the model suggests that except the age all the main effects and
their interaction effects are significant at 5% significance level
indicating their
contribution to the predictability of the model


Part (i) and Part (ii) were attempted by maximum number of students. Many of them
were successful in writing the code to fit a linear regression model but only successful
students were able to provide a good interpretation of the results. Likewise many
students were able to write the code for plotting but very few of them ended up in
writing the interpretations based on the graphs. Part (iii) was not attempted by many
students and those who attempted also ended up in writing wrong functions. Overall,
this question was very poorly done. Part (iv) was attempted well, code was written well
but a few ended up in writing correct interpretations. Part (v) was attempted by a few
students only and many of them failed to provide appropriate interpretation.

%%%%%%%%%%%%%%%%%%%%%%%
\newpage

Performance of students in CS1B varied drastically. Only a few questions were
answered successfully by majority of the participants
R code was used well by majority of the participants but they failed to make
interpretations based on the results of the code
Topics wise, students showed a good understanding of the regression models, data
analysis and visualizations but the topics on distributions, Hypothesis testing were not
addressed well.
The level of interpretation and the comments provided next to the R Code varied
significantly among the students.
A good number of students failed in submitting/pasting the output for a few R
Functions which resulted in them losing a few marks.

