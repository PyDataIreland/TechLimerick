\documentclass[a4paper,12pt]{article}
%%%%%%%%%%%%%%%%%%%%%%%%%%%%%%%%%%%%%%%%%%%%%%%%%%%%%%%%%%%%%%%%%%%%%%%%%%%%%%%%%%%%%%%%%%%%%%%%%%%%%%%%%%%%%%%%%%%%%%%%%%%%%%%%%%%%%%%%%%%%%%%%%%%%%%%%%%%%%%%%%%%%%%%%%%%%%%%%%%%%%%%%%%%%%%%%%%%%%%%%%%%%%%%%%%%%%%%%%%%%%%%%%%%%%%%%%%%%%%%%%%%%%%%%%%%%
\usepackage{eurosym}
\usepackage{vmargin}
\usepackage{amsmath}
\usepackage{graphics}
\usepackage{epsfig}
\usepackage{enumerate}
\usepackage{multicol}
\usepackage{subfigure}
\usepackage{fancyhdr}
\usepackage{listings}
\usepackage{framed}
\usepackage{graphicx}
\usepackage{amsmath}
\usepackage{chngpage}
%\usepackage{bigints}

\usepackage{vmargin}
% left top textwidth textheight headheight
% headsep footheight footskip
\setmargins{2.0cm}{2.5cm}{16 cm}{22cm}{0.5cm}{0cm}{1cm}{1cm}
\renewcommand{\baselinestretch}{1.3}

\setcounter{MaxMatrixCols}{10}

\begin{document}
\large



Fit a linear regression model to predict the “PAID” claim amount based on other variables (Consider the AGE as a numerical variable and all others as categorical).
Provide your interpretation of the model by explaining R-Squared, Adjusted R-Squared, p-value of the model and p-value of each of the coefficients. Identify the
significant variables in the prediction of “PAID” claims. 

Comment on the applicability of the linear regression model by plotting “Residuals vs. Fitted Values” and “QQ Plot of the residuals”. 

Solution 3:
i)
```{r}
#Fitting a Linear Regression Model
model1<-lm(PAID~.,data = AutoClaims)
summary(model1)
```

##
## Call:
## lm(formula = PAID ~ ., data = AutoClaims)
##
## Residuals:
## Min 1Q Median 3Q Max
## -10462 -2276 119 1611 36377
##
## Coefficients:
##
Estimate Std. Error t value Pr(>|t|)
## (Intercept) 19818.12 1391.58 14.242 < 2e-16 ***
## STATESTATE 02 -2306.41 658.69 -3.502 0.000477 ***
## STATESTATE 03 -580.09 761.36 -0.762 0.446242
## STATESTATE 04 -689.08 702.04 -0.982 0.326495
## STATESTATE 06 440.79 752.27 0.586 0.558010
## STATESTATE 07 -1254.29 837.22 -1.498 0.134318
## STATESTATE 10 2275.25 885.44 2.570 0.010284 *
## STATESTATE 12 -752.99 850.10 -0.886 0.375897
## STATESTATE 14 -404.69 842.90 -0.480 0.631216
## STATESTATE 15 -4791.86 623.56 -7.685 2.87e-14 ***
## STATESTATE 17 -883.67 704.58 -1.254 0.209982
## CLASSC11 -11743.95 430.60 -27.274 < 2e-16 ***
## CLASSC6
-14833.37 410.84 -36.105 < 2e-16 ***
## CLASSF6
-225.16 517.68 -0.435 0.663670
## GENDERM
-1193.01 215.50 -5.536 3.69e-08 ***
## AGE
15.76 24.10 0.654 0.513418
## ---
## Signif. codes: 0 '***' 0.001 '**' 0.01 '*' 0.05 '.' 0.1 ' ' 1
##
## Residual standard error: 3943 on 1401 degrees of freedom
## Multiple R-squared: 0.6886, Adjusted R-squared: 0.6852
## F-statistic: 206.5 on 15 and 1401 DF, p-value: < 2.2e-16
Note: 1 Mark can be deducted if the output is not pasted
anova(model1)
Analysis of Variance Table
Response: PAID
Df Sum Sq Mean Sq F value Pr(>F)
STATE
10 9.8027e+09 9.8027e+08 63.0629 < 2.2e-16 ***
CLASS
3 3.7869e+10 1.2623e+10 812.0763 < 2.2e-16 ***
GENDER
1 4.7271e+08 4.7271e+08 30.4106 4.158e-08 ***
AGE
1 6.6423e+06 6.6423e+06 0.4273 0.5134
Residuals 1401 2.1778e+10 1.5544e+07
---
Signif. codes: 0 ‘***’ 0.001 ‘**’ 0.01 ‘*’ 0.05 ‘.’ 0.1 ‘ ’ 1
Interpretation
R-Squared: 68.86% of the variation in the claims paid is explained by state, rating class,
gender and age

Adjusted R-Squared: 68.52% is used to compare with other models, adjusts for the number of terms in the model. We Use adjusted R-squared to compare the goodness-of-fit for regression models that contain differing numbers of independent variables.


p-value of the model is <2.2*E-16 which is less than 0.05 and hence the null hypothesis of “There is no significant linear relationship between the given independent variables X and a dependent variable Y” is rejected at 5% level of significance. Using this model to predict the DV is better than simply using the expected value of the DV as a predictor for the DV

p-value of the coefficients: While the model is overall significant, some of the variables may be insignificant. As state 1, Rating class C1 and Gender female are taken as based states and
their coefficients are clubbed in the intercept itself, we observe that coefficients of State 2 and state 15 (Negative) and State 10 (Positive) are significantly different from state 1 (At
95% Confidence level). 

Similarly rating classes C11 and C6 have significantly negative coefficients compared to C1 indicating that the claim paid for those two rating classes is
significantly lesser compared to that of C1. Males have significantly lesser claim paid compared to females at 95% confidence level

From the ANOVA table, we can infer that except Age, all other variables are significant in prediction of claims paid


%%%%%%%%%%%%%%%%%%%%%%%%%%%%%%%%%%%%%%%%%%5
\newpage 

## Exercise

ii)
```{r}
#Plot of residuals vs. Fitted Values
plot(model1$fitted.values,model1$residuals)
```
The plot is used to detect non-linearity, unequal error variances, and outliers.
The residuals "do not bounce randomly" around the 0 line. This suggests that the assumption that the relationship is linear is not reasonable.
The residuals do not form a "horizontal band" around the 0 line. This suggests that the variances of the error terms are not equal and exhibit heteroscedasticity
A few residuals "stands out" from the basic random pattern of residuals. This suggests that there are outliers.

```{r}
# QQ Plot of the residuals
qqnorm(model1$residuals)
```

A Q-Q plot is a scatterplot created by plotting two sets of quantiles against one another. 
If both sets of quantiles came from the same distribution, we should see the points forming a line that’s roughly straight. Here it is not, indicating deviance of the residuals from
normality. Thus linear regression may not be a better fit to the data. 


\end{document}