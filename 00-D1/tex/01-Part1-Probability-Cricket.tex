\documentclass[a4paper,12pt]{article}
%%%%%%%%%%%%%%%%%%%%%%%%%%%%%%%%%%%%%%%%%%%%%%%%%%%%%%%%%%%%%%%%%%%%%%%%%%%%%%%%%%%%%%%%%%%%%%%%%%%%%%%%%%%%%%%%%%%%%%%%%%%%%%%%%%%%%%%%%%%%%%%%%%%%%%%%%%%%%%%%%%%%%%%%%%%%%%%%%%%%%%%%%%%%%%%%%%%%%%%%%%%%%%%%%%%%%%%%%%%%%%%%%%%%%%%%%%%%%%%%%%%%%%%%%%%%
\usepackage{eurosym}
\usepackage{vmargin}
\usepackage{amsmath}
\usepackage{graphics}
\usepackage{epsfig}
\usepackage{enumerate}
\usepackage{multicol}
\usepackage{subfigure}
\usepackage{fancyhdr}
\usepackage{listings}
\usepackage{framed}
\usepackage{graphicx}
\usepackage{amsmath}
\usepackage{chngpage}
%\usepackage{bigints}

\usepackage{vmargin}
% left top textwidth textheight headheight
% headsep footheight footskip
\setmargins{2.0cm}{2.5cm}{16 cm}{22cm}{0.5cm}{0cm}{1cm}{1cm}
\renewcommand{\baselinestretch}{1.3}

\setcounter{MaxMatrixCols}{10}

\begin{document}


All questions need to be answered using the ‘R’ software, unless otherwise mentioned.
Q. 1)
The probability that India will win a cricket match against South Africa is 0.7
i) Prepare a probability distribution table of number of wins if Indians are going to play 10
matches against the South Africans. (4)
ii) Plot a bar chart of the probabilities of number of wins from 0 to 10. (4)
iii) Find the mean and median number of wins for India against South Africa.


Solution 1:
i)
p<-0.7
probability distribution table of number of number of wins
prob<-dbinom(0:10,size = 10,prob = 0.7)
probability_Densities<-data.frame(No_Successes = 0:10,prob)
probability_Densities
No_Successes
0
1
2
3
4
5
6
7
8
9
10


prob
0.0000059049
0.0001377810
0.0014467005
0.0090016920
0.0367569090
0.1029193452
0.2001209490
0.2668279320
0.2334744405
0.1210608210
0.0282475249

ii)
bar chart of the probabilities of number of wins
barplot(prob,main = "Bar Chart of Probability of Successes", xlab = "Numbe
r of Successes", names.arg = 0:10)

iii)
mean and median number of wins for India against South Africa
mean<-10*0.7 #or
Page 2 of 13IAI

mean<-sum(probability_Densities$No_Successes*probability_Densities$prob
)
mean

##  7

median<-qbinom(0.5,size = 10,prob = 0.7)
median

##  7




