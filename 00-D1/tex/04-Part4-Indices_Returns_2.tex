\documentclass[a4paper,12pt]{article}
%%%%%%%%%%%%%%%%%%%%%%%%%%%%%%%%%%%%%%%%%%%%%%%%%%%%%%%%%%%%%%%%%%%%%%%%%%%%%%%%%%%%%%%%%%%%%%%%%%%%%%%%%%%%%%%%%%%%%%%%%%%%%%%%%%%%%%%%%%%%%%%%%%%%%%%%%%%%%%%%%%%%%%%%%%%%%%%%%%%%%%%%%%%%%%%%%%%%%%%%%%%%%%%%%%%%%%%%%%%%%%%%%%%%%%%%%%%%%%%%%%%%%%%%%%%%
\usepackage{eurosym}
\usepackage{vmargin}
\usepackage{amsmath}
\usepackage{graphics}
\usepackage{epsfig}
\usepackage{enumerate}
\usepackage{multicol}
\usepackage{subfigure}
\usepackage{fancyhdr}
\usepackage{listings}
\usepackage{framed}
\usepackage{graphicx}
\usepackage{amsmath}
\usepackage{chngpage}
%\usepackage{bigints}

\usepackage{vmargin}
% left top textwidth textheight headheight
% headsep footheight footskip
\setmargins{2.0cm}{2.5cm}{16 cm}{22cm}{0.5cm}{0cm}{1cm}{1cm}
\renewcommand{\baselinestretch}{1.3}

\setcounter{MaxMatrixCols}{10}

\begin{document}


\large 
\noindent Compute the 5-th percentile, 1st quartile, median, 3rd quartile and 95 th percentile of both
the actual claims paid as well as from the fitted distributions. 




iii)
#Computed quantiles of the actual data as well as that of different distributions

\begin{framed}
\begin{verbatim}
quantile(AutoClaims$PAID,c(0.05,0.25,0.5,0.75,0.95))
\end{verbatim}
\end{framed}


%%%%%%%%%%%%%%%%%
\newpage 
\noindent Using the results from (ii) and (iii), comment on goodness-of-fit of the models to the
data.


\begin{framed}
\begin{verbatim}
##
5%
25%
50%
75%
95%
## 1116.379 1610.660 3395.368 7774.382 20405.050
qnorm(c(0.05,0.25,0.5,0.75,0.95),mean = Normalmu,sd = Normalsigma)

##  -5431.878 1387.290 6127.222 10867.155 17686.323
qlnorm(c(0.05,0.25,0.5,0.75,0.95),meanlog = LNmu,sdlog = LNsigma)

##  892.0584 2170.4062 4026.6904 7470.5996 18176.2032
qexp(c(0.05,0.25,0.5,0.75,0.95),rate = Exprate)

##  314.2854 1762.6920 4247.0670 8494.1339 18355.5180
qgamma(c(0.05,0.25,0.5,0.75,0.95),shape = GammaAlpha,rate = GammaBeta)

##  142.0733 1276.6130 3737.9364 8453.3371 20244.3595
\end{verbatim}
\end{framed}
%%%%%%%%%%%%%%%%%%%%%%%%%%%%%%%%%%%%%%%%%%%%%%%%
iv)
#Comment based on (ii) and (iii)


\begin{itemize}
    \item From the histogram and the superimposed plots, it is clear that normal distribution does not
fit the data well.

The other three curves are getting superimposed more or less similarly to the data. 
\item Even from the quantiles we observe that lower values are aptly modeled using lognormal distribution (5 th percentile of lognormal being close to actual values) whereas gamma distribution is modeling the higher values more appropriately (95 th percentile).
\item Hence, just by looking at (ii) and (iii), best fitting distribution among Lognormal, Gamma and
Exponential distributions cannot be concluded. 
\item t requires additional analysis in the form of other statistical tests to confirm the best fit
\end{itemize}





%%%%%%%%%%%%%%%%%%%%%%%%%%%%%%%%%%%%%%%%%%%%
\newpage 
Part (i) was well attempted though many students had difficulty in arriving at the parameter
s corresponding to lognormal distribution and Gamma distribution. 

All those who attempted
Part (i) successfully were able to attempt Part (ii) and Part (iii) as well, but the failure in fittin
g a few distributions in part (i) resulted in only part answers for part (ii) and (iii). Many stude
nts were able to plot the histogram and the distributions decently well but the labelling thro
ugh appropriate legends was not done well. 

Part (iv) was not attempted by many students a
nd among them who attempted, complete interpretation was lagging. Very few wrote a det
ailed interpretation of the result.

\end{document}
