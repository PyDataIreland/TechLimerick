\documentclass[a4paper,12pt]{article}
%%%%%%%%%%%%%%%%%%%%%%%%%%%%%%%%%%%%%%%%%%%%%%%%%%%%%%%%%%%%%%%%%%%%%%%%%%%%%%%%%%%%%%%%%%%%%%%%%%%%%%%%%%%%%%%%%%%%%%%%%%%%%%%%%%%%%%%%%%%%%%%%%%%%%%%%%%%%%%%%%%%%%%%%%%%%%%%%%%%%%%%%%%%%%%%%%%%%%%%%%%%%%%%%%%%%%%%%%%%%%%%%%%%%%%%%%%%%%%%%%%%%%%%%%%%%
\usepackage{eurosym}
\usepackage{vmargin}
\usepackage{amsmath}
\usepackage{graphics}
\usepackage{epsfig}
\usepackage{enumerate}
\usepackage{multicol}
\usepackage{subfigure}
\usepackage{fancyhdr}
\usepackage{listings}
\usepackage{framed}
\usepackage{graphicx}
\usepackage{amsmath}
\usepackage{chngpage}
%\usepackage{bigints}

\usepackage{vmargin}
% left top textwidth textheight headheight
% headsep footheight footskip
\setmargins{2.0cm}{2.5cm}{16 cm}{22cm}{0.5cm}{0cm}{1cm}{1cm}
\renewcommand{\baselinestretch}{1.3}

\setcounter{MaxMatrixCols}{10}

\begin{document}
Q. 2) ABC General Insurance Company Limited has sold total of 1000 insurance policies under 3
type of insurance, where each type of insurance gives rise to a maximum of one claim per year.
The insurer hold the following portfolio:
Number of
Policies
300
500
200
Portfolio
Fire Insurance
Marine Cargo Insurance
Marine Hull Insurance
Claim
distribution
Exp (0.006)
Exp (0.007)
Exp (0.006)
Probability of
claim
0.002
0.001
0.003
All policies and claims are independent.
i)
ii)
Simulate a set of 5,000 observations from the insurer’s aggregate claims size
distribution. Use seed corresponding to your birth year.
Summarize the output, draw plot histogram of the simulated results, and comment on
the results.
(12)
(5)


Solution 2:
i) set.seed(2000)
rate1 <- 0.006
q1 <- 0.002
n1 <- 300
rate2 <- 0.007
q2 <- 0.001
n2 <- 500
rate3 <- 0.006
q3 <- 0.003
n3 <- 200
Sumofclaims_sim <- numeric(5000)
Total_policies = n1+n2+n3
for (j in 1:5000){
x<-numeric(Total_policies)
for (i in 1:Total_policies)
{if (i<= n1){rate=rate1} else{if(i<=n1+n2){rate=rate2}else{rate=rate3}}
x[i] <- rexp(1,rate=rate)}
death<-numeric(1000)
for (i in 1:Total_policies)
{if (i<= n1){prob=q1} else{if(i<=n1+n2){prob=q2}else{prob=q3}}
death[i] <- rbinom(1,size=1,prob)}
sim_claim <- x*death
sim_claim<-sum(sim_claim)
Sumofclaims_sim[j] <- sim_claim
}
Sumofclaims_sim
[4981]
[4987]
[4993]
[4999]
48.0509705 180.6917655 387.8498483 158.5486544 79.6699608 718.4442927
260.0237473 58.7869147 0.0000000 824.1810537 121.5688946 393.4891939
34.2667969 0.0000000 0.0000000 323.9416166 0.0000000 47.5719195

%%%%%%%%%%%%%%%%%%%%%%%%%%%%%%%%%%%%%%%%%%%%%%%%%%%%%%%%%%%%%%%%%%%



%%%%%%%%%%%%%%%%%%%%%%%%%%%%%%%%%%%%%%%%%%%%%%%%%%%%%%%%%%%%%%%%%%%
400.2364413 155.3775077
ii) summary(Sumofclaims_sim)
hist(Sumofclaims_sim)

%---------------------------%
> summary(Sumofclaims_sim)
Min. 1st Qu. Median
Mean 3rd Qu.
Max.
0.00
29.03 175.60 260.88 393.86 2664.99
[Note the numbers will change with change in seed]
The claims seem to be highly skewed with Q1 equal to 29.03, Q2 equal to 260.88 and Q3 equal to
393.86. The highest value is 2664.99.

The distribution seems to follow exponential distribution, which is consistent with underlying
distribution of claims.


\end{document}

