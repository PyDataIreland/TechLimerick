\documentclass[a4paper,12pt]{article}
%%%%%%%%%%%%%%%%%%%%%%%%%%%%%%%%%%%%%%%%%%%%%%%%%%%%%%%%%%%%%%%%%%%%%%%%%%%%%%%%%%%%%%%%%%%%%%%%%%%%%%%%%%%%%%%%%%%%%%%%%%%%%%%%%%%%%%%%%%%%%%%%%%%%%%%%%%%%%%%%%%%%%%%%%%%%%%%%%%%%%%%%%%%%%%%%%%%%%%%%%%%%%%%%%%%%%%%%%%%%%%%%%%%%%%%%%%%%%%%%%%%%%%%%%%%%
\usepackage{eurosym}
\usepackage{vmargin}
\usepackage{amsmath}
\usepackage{graphics}
\usepackage{epsfig}
\usepackage{enumerate}
\usepackage{multicol}
\usepackage{subfigure}
\usepackage{fancyhdr}
\usepackage{listings}
\usepackage{framed}
\usepackage{graphicx}
\usepackage{amsmath}
\usepackage{chngpage}
%\usepackage{bigints}

\usepackage{vmargin}
% left top textwidth textheight headheight
% headsep footheight footskip
\setmargins{2.0cm}{2.5cm}{16 cm}{22cm}{0.5cm}{0cm}{1cm}{1cm}
\renewcommand{\baselinestretch}{1.3}

\setcounter{MaxMatrixCols}{10}

\begin{document}

\large 
\noindent

2 A researcher has collected the following data on a group of students, regarding
whether they passed or failed an exam and whether or not they attended tutorials:
Number of students Exam passed Exam failed
Attended tutorials 132 27
Did not attend tutorials 120 51
(i) State the hypotheses of this test. [2]
[Total 11]
The data can be entered into R using the following code:
amounts=c(1.95,1.80,2.10,1.82,1.75,2.01,1.83,1.90)
The data can be entered into R in matrix form using the following code:
The reasercher wants to establish whether tutorial attendance is independent of exam
success, using a chi-square test.
(ii) Calculate the expected frequencies for the data under the null hypotheses
in part (i). [3]
(iii) Perform the test stating clearly your conclusion. [6]
exam.success = matrix(c(132,120,27,51),ncol=2,nrow=2)
CS1B S2020–3