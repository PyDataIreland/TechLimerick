\documentclass[a4paper,12pt]{article}
%%%%%%%%%%%%%%%%%%%%%%%%%%%%%%%%%%%%%%%%%%%%%%%%%%%%%%%%%%%%%%%%%%%%%%%%%%%%%%%%%%%%%%%%%%%%%%%%%%%%%%%%%%%%%%%%%%%%%%%%%%%%%%%%%%%%%%%%%%%%%%%%%%%%%%%%%%%%%%%%%%%%%%%%%%%%%%%%%%%%%%%%%%%%%%%%%%%%%%%%%%%%%%%%%%%%%%%%%%%%%%%%%%%%%%%%%%%%%%%%%%%%%%%%%%%%
\usepackage{eurosym}
\usepackage{vmargin}
\usepackage{amsmath}
\usepackage{graphics}
\usepackage{epsfig}
\usepackage{enumerate}
\usepackage{multicol}
\usepackage{subfigure}
\usepackage{fancyhdr}
\usepackage{listings}
\usepackage{framed}
\usepackage{graphicx}
\usepackage{amsmath}
\usepackage{chngpage}
%\usepackage{bigints}

\usepackage{vmargin}
% left top textwidth textheight headheight
% headsep footheight footskip
\setmargins{2.0cm}{2.5cm}{16 cm}{22cm}{0.5cm}{0cm}{1cm}{1cm}
\renewcommand{\baselinestretch}{1.3}

\setcounter{MaxMatrixCols}{10}

\begin{document}

\large 
\noindent

5 A waiting time random variable X follows an Exponential distribution with rate l
parameterised as f (x) = le–lx (x > 0, l > 0).
The rate l has a Gamma prior distribution with parameters a and s. A Bayesian
credibility model provides that the posterior mean of 1/l can be expressed as
Z3x + (1– Z)3
s
α – 1
, where Z = n
α + n – 1
with n being the number of past waiting times observed.
Assume that the parameters of the prior Gamma distribution are a = 5 and s = 1.
(i) Determine an estimate of the posterior mean of 1/l assuming n = 10 by
implementing M = 3,000 Monte Carlo repetitions. [14]
(ii) Determine an estimate of the posterior mean of 1/l and the value of x when
n = 1,000, by implementing M = 3,000 Monte Carlo repetitions. [15]
(iii) Plot the histograms of the samples of the posterior mean of 1/l and of x
obtained in part (ii). [4]
(iv) Compare, by visual inspection of the graphs in part (iii), the distributions
of the posterior mean of 1/l and the distribution of x when n = 1,000. [2]
(v) Comment on the behaviour of the credibility model as n increases, relating
your answer to your findings in part (iv). [4]
[Total 39]
END OF PAPER