\documentclass[a4paper,12pt]{article}
%%%%%%%%%%%%%%%%%%%%%%%%%%%%%%%%%%%%%%%%%%%%%%%%%%%%%%%%%%%%%%%%%%%%%%%%%%%%%%%%%%%%%%%%%%%%%%%%%%%%%%%%%%%%%%%%%%%%%%%%%%%%%%%%%%%%%%%%%%%%%%%%%%%%%%%%%%%%%%%%%%%%%%%%%%%%%%%%%%%%%%%%%%%%%%%%%%%%%%%%%%%%%%%%%%%%%%%%%%%%%%%%%%%%%%%%%%%%%%%%%%%%%%%%%%%%
\usepackage{eurosym}
\usepackage{vmargin}
\usepackage{amsmath}
\usepackage{graphics}
\usepackage{epsfig}
\usepackage{enumerate}
\usepackage{multicol}
\usepackage{subfigure}
\usepackage{fancyhdr}
\usepackage{listings}
\usepackage{framed}
\usepackage{graphicx}
\usepackage{amsmath}
\usepackage{chngpage}
%\usepackage{bigints}

\usepackage{vmargin}
% left top textwidth textheight headheight
% headsep footheight footskip
\setmargins{2.0cm}{2.5cm}{16 cm}{22cm}{0.5cm}{0cm}{1cm}{1cm}
\renewcommand{\baselinestretch}{1.3}

\setcounter{MaxMatrixCols}{10}

\begin{document}

\large 
\noindent

4 Data were collected on average alcohol and cigarette consumption per adult per year
for a number of countries. The data are given in the file smoking_data.RData
and contain the following information:
country: the country concerned;
alcohol: alcohol consumption per adult per year (litres/year);
cigarettes: number of cigarettes consumed per adult per year.
(i) (a) Construct a scatterplot of the data with alcohol consumption on the
x axis.
(b) Comment on the relationship between the data on alcohol and cigarette
consumption based on your plot in part (i)(a).
[5]
(ii) Calculate Pearson’s correlation coefficient between the data on alcohol and
cigarette consumption. [2]
An analyst suggests using the following R code to modify the data:
(iii) Explain what the above code does and give a justification for its use. [3]
For the remainder of the question, use the modified data
(alcohol.2, cigarettes.2), as produced by applying the R
code above.
(iv) (a) Construct a scatterplot with alcohol consumption on the x axis.
(b) Calculate Pearson’s correlation coefficient between the new data on
alcohol and cigarette consumption.
(c) Comment on your answers in parts (ii) and (iv)(b).
[6]
(v) Perform a hypothesis test for the null hypothesis that Pearson’s population
correlation coefficient is equal to zero, against the alternative that it is positive.
You should report the p-value of the test and a clear conclusion. [5]
A media bulletin has reported that ‘'‘h‘igher alcohol consumption causes higher
cigarette consumption .
(vi) Comment on whether this report is justified based on your analysis in parts (iv)
and (v). [2]
‘
’
alcohol.2 = alcohol[-c(6,16)]
cigarettes.2 = cigarettes[-c(6,16)]
CS1B S2020–5
(vii) Perform a simple linear regression analysis on the new data using a
model of the form y = α + βx + ε (cigarette consumption, y, on alcohol
consumption, x), where the error terms ε independently follow a
N(0, s2) distribution.
Your answer should show the fitted line plotted on the data scatterplot
and report the estimate of parameter s. [5]
(viii) State the proportion of the total variability of the responses explained
by the model, based on your output in part (vii). [1]
(ix) Plot a graph of the residuals of the model fitted in part (vii)
against the explanatory variable. [2]
(x) Comment on the validity of the model, based on your output in
part (ix) . [3]
[Total 34]
CS1B S2020–6