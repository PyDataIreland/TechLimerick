\documentclass[a4paper,12pt]{article}
%%%%%%%%%%%%%%%%%%%%%%%%%%%%%%%%%%%%%%%%%%%%%%%%%%%%%%%%%%%%%%%%%%%%%%%%%%%%%%%%%%%%%%%%%%%%%%%%%%%%%%%%%%%%%%%%%%%%%%%%%%%%%%%%%%%%%%%%%%%%%%%%%%%%%%%%%%%%%%%%%%%%%%%%%%%%%%%%%%%%%%%%%%%%%%%%%%%%%%%%%%%%%%%%%%%%%%%%%%%%%%%%%%%%%%%%%%%%%%%%%%%%%%%%%%%%
\usepackage{eurosym}
\usepackage{vmargin}
\usepackage{amsmath}
\usepackage{graphics}
\usepackage{epsfig}
\usepackage{enumerate}
\usepackage{multicol}
\usepackage{subfigure}
\usepackage{fancyhdr}
\usepackage{listings}
\usepackage{framed}
\usepackage{graphicx}
\usepackage{amsmath}
\usepackage{chngpage}
%\usepackage{bigints}

\usepackage{vmargin}
% left top textwidth textheight headheight
% headsep footheight footskip
\setmargins{2.0cm}{2.5cm}{16 cm}{22cm}{0.5cm}{0cm}{1cm}{1cm}
\renewcommand{\baselinestretch}{1.3}

\setcounter{MaxMatrixCols}{10}

\begin{document}

\large 
\noindent

3 A machine in a sweet factory fills bags of sweets to weigh 500 grams. The actual
weight of the sweet bags is known to follow a Normal distribution. The sweet
manufacturer believes that the machine is under-filling the sweet bags. A sample of
10 sweet bags is taken and weighed, as summarised below.
Bag 1 2 3 4 5 6 7 8 9 10
Weight (grams) 474.11 512.01 493.64 495.03 518.13 486.03 494.48 501.76 498.83 503.02
The data can be entered into R using the following code:
weight=c(474.11,512.01,493.64,495.03,518.13,486.03,
494.48,501.76,498.83,503.02)
(i) Perform a suitable t-test to determine whether the sweet bags are being
consistently under-filled, stating the hypotheses and the level of significance
used in the test. [8]
(ii) Propose an interpretation of your conclusion in part (i). [2]
[Total 10]
CS1B S2020–4