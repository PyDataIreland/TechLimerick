\documentclass[a4paper,12pt]{article}
%%%%%%%%%%%%%%%%%%%%%%%%%%%%%%%%%%%%%%%%%%%%%%%%%%%%%%%%%%%%%%%%%%%%%%%%%%%%%%%%%%%%%%%%%%%%%%%%%%%%%%%%%%%%%%%%%%%%%%%%%%%%%%%%%%%%%%%%%%%%%%%%%%%%%%%%%%%%%%%%%%%%%%%%%%%%%%%%%%%%%%%%%%%%%%%%%%%%%%%%%%%%%%%%%%%%%%%%%%%%%%%%%%%%%%%%%%%%%%%%%%%%%%%%%%%%
\usepackage{eurosym}
\usepackage{vmargin}
\usepackage{amsmath}
\usepackage{graphics}
\usepackage{epsfig}
\usepackage{enumerate}
\usepackage{multicol}
\usepackage{subfigure}
\usepackage{fancyhdr}
\usepackage{listings}
\usepackage{framed}
\usepackage{graphicx}
\usepackage{amsmath}
\usepackage{chngpage}
%\usepackage{bigints}

\usepackage{vmargin}
% left top textwidth textheight headheight
% headsep footheight footskip
\setmargins{2.0cm}{2.5cm}{16 cm}{22cm}{0.5cm}{0cm}{1cm}{1cm}
\renewcommand{\baselinestretch}{1.3}

\setcounter{MaxMatrixCols}{10}

\begin{document}


\section{Introduction}
\begin{enumerate}
\item The following information is available on the returns on various assets under the three possible states of the world, each of which has an equal probability of occurring: 
\begin{center}
 \begin{tabular}{c|c|c|c}
  & 1 & 2 & 3 \\ 
Share A &  -8\% & -20\% &  62\%\\ 
Share B &   30\% & -20\% & 37\% \\
Market Portfolio & 45\% & -15\% & 20\%\\ 
Risk-Free Asset & 10\% & 10\% & 10\%\\ 
\end{tabular}   
\end{center}


\begin{enumerate}[(i)]
    \item What is the numerical equation for the Capital Market Line (CML) for this stock market?  
\item Do shares A and B lie on the CML, and what, if anything do you conclude from this? (7 marks) 
\item What are the beta values for shares A and B? (6 marks) 
\item Are the expected returns for shares A and B consistent with the Security Market Line, and, if not, what equilibrating changes would you expect to occur? (7 marks) 
\end{enumerate}

\item Assume that stock returns are generated by a two-factor model. The returns on three well diversified portfolios, A, B, and C, are given by the following representations: 
\begin{eqnarray*}
r_A &=& 0.10 + 2Fi F2 \\
r_B &=& 0.08 + 4Fi - 2F2 \\
r_c &=& 0.05 + - F2 \\
\end{eqnarray*}
\begin{enumerate}[(i)]
    \item Discuss what the factor representations above imply for the variation and comovement in the three stock returns  
\item Find the portfolio weights that one must place on stocks A, B, and C to construct pure tracking portfolios for the two factors (i.e. portfolios in which the loading on the relevant factor is +1 and the loadings on all other factors are 0). (8 marks) 
\item  If one was to introduce a new portfolio, D, with loadings of +2 on both of the factors, what would the expected return on D have to be to rule out arbitrage?  
\item  Use your results from previous parts of the questions to give a short description of the workings of the Arbitrage Pricing Theory. (7 marks) 
\end{enumerate}
\item Denote the price of a European call option written on a non-dividend paying stock S with exercise price X maturing in T periods by $c(S,X,T)$: 
\begin{enumerate}[(i)]
    \item  Derive a simple upper bound on $c(S,X,T)$. (4 marks) 
\item  Present an arbitrage argument to demonstrate that the following lower bound must hold, $$c(S,X,T)>max [0, S-Xen$$, where $r$ denotes the risk-free rate. 
\item  Present an arbitrage argument to demonstrate that $c(S,X,T) + c(S,3X,T)$ $>$ $2 c(S, 2X,T)$. 
\end{enumerate}
\item Denote by c the price of a European call option and p the price of a European put option written on a non-dividend paying stock. Denote the current stock price with $S$ and assume that the exercise price of both options is $X$. The risk-free rate is r and both options mature in T periods. 
\begin{enumerate}[(i)]
    \item  Derive simple upper bounds for both the put and call price in terms of the current stock price, exercise price, risk-free rate and time to maturity of the option. What arguments underlie these bounds? 
\item Present an arbitrage argument to demonstrate that the following lower bound must hold for the European call option; 
$$c>. max[0, S Xe rT) $$
\item  Present a similar argument that leads to the following lower bound for a Euro-pean put; 
p>max[0,Xe rT 
\item  Present an argument which leads to the result that it is never optimal to exercise an American call option on a non-dividend paying stock before the maturity date.  
\end{enumerate}
\item Consider a firm that only lasts one period (from t=0 to t=1). The firm has currently 60 units of cash. The firm also has existing assets that will produce 60 units of cash at t=1 if the economy is booming or 0 units of cash if the economy is in crisis. The probability of the economy booming is 0.5 (and the probability of the economy being in crisis is therefore 0.5 too) The firm also has debt with face value 80 that matures at t=1. All agents are risk neutral and the discount rate is zero. 
\begin{enumerate}
    \item (a) Calculate the value at t=0 of the equity and debt of the firm as well as the firm's total value.  
    \item  Suppose that the firm faces 2 mutually exclusive projects. Project A costs 40 units at t=0 and produces 60 units at t=1 for sure. Project B costs 60 units at t=0 and produces 100 units at t=1 if the economy is booming or 0 units if the economy is in crisis. These projects are mutually exclusive. Which project would the shareholders of the firm chose? Calculate again the value of debt equity and whole firm once this project has been undertaken. What project would be undertaken if the firm initially had zero debt? What would be the total value of the firm in this case?  
    \item In the spirit of the Jensen and Meckling (1976) paper, explain how the way that a firm uses to raise external finance may affect the effort levels provided by the entrepreneur-manager of the firm  
    \item In light of the results of section (b) explain what is known as the asset substitution or risk shifting effect. How would the optimal level of debt of a firm be determined according to the effects shown in parts (b) and (c)?  
\end{enumerate}


\item  Dividend policy: 
(a) Carefully explain what were the stylised facts about dividend policy that John Lintner identified after interviewing several managers of large US corporations. How do these stylised facts contradict the predictions of the Modigliani and Miller theorem?  
\item  Dividend policy: Explain how different levels of personal taxation on capital gains and dividend income will affect the preferences of investors towards firms with alternative dividend policies. How would firms react to this situation? What would be the equilibrium of this situation? - use a numerical example if necessary to illustrate your answer.  

\item  An investor has two assets available from which to form his desired portfolio. Asset A has an expected return of 5\% and a return standard deviation of 10\%. Asset B has an expected return of 9\% and a return standard deviation of 16\%. The correlation of returns is 18\%. The investor is prohibited from selling either of the assets short. 
\begin{enumerate}[(i)]
\item Derive the minimum variance portfolio. What are its expected return 0 and standard deviation of returns?

\item Graphically depict the mean-variance frontier. What underlies this frontier shape? Are there any portfolios on the mean-variance frontier the investor would not wish to invest in regardless of his level of risk-aversion? 
\item How does the mean variance frontier change when the correlation of returns increases ? 
\item How does the mean-variance frontier change when the correlation of returns increases to 1?  
\end{enumerate}

\item Debt Overhang: Explain why a firm might forego some good projects (positive net present value) due to a "debt overhang problem". 
\item Debt Overhang: Consider a simplified model of a firm that lasts for one period (from dates $t=0$ to $t=1$). The firm has outstanding debt with face value D=80 that has to be repaid at $t=1$. The firm also has some existing projects that generate 20 units of cash if the economy is in a slump at $t=1$ and 100 units of cash if the economy is in a boom at $t-1$ The probability of the economy booming is 0.5. 
\begin{enumerate}[(i)]
    \item What is the market value of the debt of this firm? [5 marks]
    \item Would the firm undertake a safe project financed by equity that costs 40 units at t-0 and produces 60 units for sure at $t-1$?  
    \item Would the project be undertaken if the face value of outstanding debt was 40? What is the key difference between the situation in part ii) and part iii) ?  
\end{enumerate}
\item Dividend Policy \\ 
Briefly list and discuss Lintner's empirical results on the dividend policy of US corporations. 
\item Dividend Policy \\ Illustrate how dividends could be used to signal the quality of a firm. State the necessary elements that a model of signalling through dividends should have. 

\end{enumerate}
\end{document}
