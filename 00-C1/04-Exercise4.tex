

```R
library(magrittr)
```

A recent study suggests that the maximum heart rate of a person, related to age in
years, is given by the equation:

$$ \mbox{Max Rate} = 220 – \mbox{Age}$$

Suppose this is to be empirically proven and 15 people of varying ages are tested for
their maximum heart rate. 



The following data are collected:

| | | | | | | | | | | | |
|:-------------:|------|------|------|------|------|------|------|------|------|------|
|***Age (years)***| 18| 23| 25| 35| 65| 54| 34| 56| 72| 19| 23| 42| 18| 39| 37|
|***Max Rate ***| 202| 186| 187| 180| 156| 169| 174| 172| 153| 199| 193| 174| 198| 183| 178|

The data can be entered into R using the following commands:

<pre><code>
x &lt;- c(18,23,25,35,65,54,34,56,72,19,23,42,18,39,37)
y &lt;- c(202,186,187,180,156,169,174,172,153,199,193,174, 198,183,178)
</code></pre>

## Exercise 1

Plot the fitted line for the regression of Max Rate on Age. 



```R
x <- c(18,23,25,35,65,54,34,56,72,19,23,42,18,39,37)

y <- c(202,186,187,180,156,169,174,172,153,199,193,174,198,183,178)

plot(x,y)
```


![png](output_4_0.png)



```R
plot(x,y,pch=18)
```


![png](output_5_0.png)



```R
lm.result = lm(y ~ x)
```


```R
# Obtain the basic values of the regression analysis

lm.result
```


    
    Call:
    lm(formula = y ~ x)
    
    Coefficients:
    (Intercept)            x  
       210.0485      -0.7977  




```R
# make a plot
plot(x,y,pch=18)

# plot the regression line
abline(lm(y ~ x)) 


```


![png](output_8_0.png)



```R
summary(lm.result)
```


    
    Call:
    lm(formula = y ~ x)
    
    Residuals:
        Min      1Q  Median      3Q     Max 
    -8.9258 -2.5383  0.3879  3.1867  6.6242 
    
    Coefficients:
                 Estimate Std. Error t value Pr(>|t|)    
    (Intercept) 210.04846    2.86694   73.27  < 2e-16 ***
    x            -0.79773    0.06996  -11.40 3.85e-08 ***
    ---
    Signif. codes:  0 ‘***’ 0.001 ‘**’ 0.01 ‘*’ 0.05 ‘.’ 0.1 ‘ ’ 1
    
    Residual standard error: 4.578 on 13 degrees of freedom
    Multiple R-squared:  0.9091,	Adjusted R-squared:  0.9021 
    F-statistic:   130 on 1 and 13 DF,  p-value: 3.848e-08



Max heart rate reduces as age increases. The fit of the model seems good.

## Exercise 2

Comment on the results.

A researcher reviews the plot in ***Exercise 1*** and suggests the slope should be equal to –1.


```R
confint(lm.result) %>% round(2)
```


<table>
<caption>A matrix: 2 x 2 of type dbl</caption>
<thead>
	<tr><th></th><th>2.5 %</th><th>97.5 %</th></tr>
</thead>
<tbody>
	<tr><th>(Intercept)</th><td>203.85</td><td>216.24</td></tr>
	<tr><th>x</th><td> -0.95</td><td> -0.65</td></tr>
</tbody>
</table>



## Exercise 3

Calculate the p-value of a hypothesis test for this suggestion, by creating a suitable test statistic.


* Can do a test to see if the slope of -1 is correct. 


* Let $H_0$ be that $\beta = -1$, and $H_A$ be that $\beta \neq -1 $. 


* Then we can create the test statistic and the p-value as follows:



```R
es <- resid(lm.result)
# the residuals lm.result

es %>% round(3)
```


<dl class="dl-horizontal">
	<dt>1</dt>
		<dd>6.311</dd>
	<dt>2</dt>
		<dd>-5.701</dd>
	<dt>3</dt>
		<dd>-3.105</dd>
	<dt>4</dt>
		<dd>-2.128</dd>
	<dt>5</dt>
		<dd>-2.196</dd>
	<dt>6</dt>
		<dd>2.029</dd>
	<dt>7</dt>
		<dd>-8.926</dd>
	<dt>8</dt>
		<dd>6.624</dd>
	<dt>9</dt>
		<dd>0.388</dd>
	<dt>10</dt>
		<dd>4.108</dd>
	<dt>11</dt>
		<dd>1.299</dd>
	<dt>12</dt>
		<dd>-2.544</dd>
	<dt>13</dt>
		<dd>2.311</dd>
	<dt>14</dt>
		<dd>4.063</dd>
	<dt>15</dt>
		<dd>-2.533</dd>
</dl>




```R
b1 <- (coef(lm.result))[['x']] #the x part of the coefficients
n  <-  15

b1 %>% round(3)

n
```


-0.798



15



```R
s  <-  sqrt( sum( es^2 ) / (n-2) )
SE <-  s/sqrt(sum((x-mean(x))^2))
```


```R
t  <-  (b1 - (-1) )/SE
t
```

* find the right tail for this value of t with 15-2 d.f.


```R
pt(t,13,lower.tail=FALSE)
```


0.00631015669476853



```R

The p-value is twice this as the problem is two-sided,



```


```R
2 * pt(t,13,lower.tail=FALSE)
```


```R
## Exercise 4
Comment on the researcher’s suggestion, using your answer to part (iii). 




The null hypothesis is rejected at the 5% level of significance. The slope may not be
equal to -1 for these data. (Which is the slope predicted by the original formula 220 -
Age.)

[Total 16]
Well answered overall. In part (i) a number of candidates did not show the fitted line on
the graph as required. In part (ii) many candidates failed to comment on the relationship
between the two variables. Part (iii) was answered well only by well prepared candidates.
In part (iv) answers using a different level of significance and consistent conclusion were
given credit as appropriate.

```
