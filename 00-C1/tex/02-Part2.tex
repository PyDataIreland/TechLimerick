\documentclass[a4paper,12pt]{article}
%%%%%%%%%%%%%%%%%%%%%%%%%%%%%%%%%%%%%%%%%%%%%%%%%%%%%%%%%%%%%%%%%%%%%%%%%%%%%%%%%%%%%%%%%%%%%%%%%%%%%%%%%%%%%%%%%%%%%%%%%%%%%%%%%%%%%%%%%%%%%%%%%%%%%%%%%%%%%%%%%%%%%%%%%%%%%%%%%%%%%%%%%%%%%%%%%%%%%%%%%%%%%%%%%%%%%%%%%%%%%%%%%%%%%%%%%%%%%%%%%%%%%%%%%%%%
\usepackage{eurosym}
\usepackage{vmargin}
\usepackage{amsmath}
\usepackage{graphics}
\usepackage{epsfig}
\usepackage{enumerate}
\usepackage{multicol}
\usepackage{subfigure}
\usepackage{fancyhdr}
\usepackage{listings}
\usepackage{framed}
\usepackage{graphicx}
\usepackage{amsmath}
\usepackage{chngpage}
%\usepackage{bigints}

\usepackage{vmargin}
% left top textwidth textheight headheight
% headsep footheight footskip
\setmargins{2.0cm}{2.5cm}{16 cm}{22cm}{0.5cm}{0cm}{1cm}{1cm}
\renewcommand{\baselinestretch}{1.3}

\setcounter{MaxMatrixCols}{10}

\begin{document}
\Large 
\noindent The dataset “\texttt{Interest\_rates.csv}” contains a series of returns on bonds of maturities 
1 year, 5 years, 10 years, 15 years, 20 years and 30 years (i.e. bonds that provide a
return of the principal investment after 1, 5, 10, 15, 20 and 30 years respectively).



\begin{framed} \begin{verbatim}
interest_rates <- read.csv("Interest_rates.csv")
\end{verbatim} \end{framed}


\newpage 


\noindent Calculate the Pearson correlation coefficient matrix for this data.
Comment on the correlations between the data using the matrix from
Exercise 1



\begin{framed} \begin{verbatim}

# read the data
> interest_rates<-read.csv("Interest_rates.csv")
# calculate the Pearson correlation coefficients

\end{verbatim} \end{framed}


\begin{framed} \begin{verbatim}
 C <- cor(interest_rates, method="pearson")
\end{verbatim} \end{framed}

Alternative solution
cor(interest_rates)
> C
X1.year
X5.year
X10.year
X15.year
X20.year
X30.year
X1.year
1.0000000
0.8760513
0.7827029
0.7380130
0.6146684
0.4094968
X5.year
0.8760513
1.0000000
0.9666006
0.9356486
0.8269608
0.6020591
X10.year
0.7827029
0.9666006
1.0000000
0.9907851
0.9339235
0.7657966
X15.year
0.7380130
0.9356486
0.9907851
1.0000000
0.9696405
0.8315513
X20.year
0.6146684
0.8269608
0.9339235
0.9696405
1.0000000
0.9379375
X30.year
0.4094968
0.6020591
0.7657966
0.8315513
0.9379375
1.0000000


\end{verbatim} \end{framed}



The correlation matrix shows that there is strong (positive) correlation between returns on bonds of similar maturity.

It also shows that the correlation between returns is weaker as the length of maturity between bonds increases.
%%%%%%%%%%%%%%%%%%%%%%%%%%%%%%%%%
\newpage 
\subsection*{Part 2}
\noindent Perform a reduction of the dimension of the data using principal component analysis with the method of singular value decomposition.

\noindent Your answer should include a summary of the principal component analysis.
\begin{framed} 
\begin{verbatim}
# carry out principal component analysis using SVD method

pca <- prcomp(interest_rates)
\end{verbatim} \end{framed}

\newpage
(b) Suggest with reasons, using the output of the R analysis, how many components of the transformed data should be retained.






\begin{framed} \begin{verbatim}



# review the results of the principal component analysis
> summary(pca)

Importance of components:
PC1
PC2
PC3
PC4
PC5
Standard deviation
0.00945 0.003419 0.001488 0.0002555 0.0002061
Proportion of Variance 0.86432 0.113130 0.021430 0.0006300 0.0004100
Cumulative Proportion 0.86432 0.977440 0.998880 0.9995100 0.9999200
PC6
Standard deviation
8.986e-05
Proportion of Variance 8.000e-05
Cumulative Proportion 1.000e+00
\end{verbatim}
\end{framed}

%%%%%%%%%%%%%%%%%%%%%%%%%%%%%%%%%%%%%%%%%%%
\newpage 

\begin{framed}
\begin{verbatim}

> pca<-princomp(interest_rates)
> summary(pca)
Importance of components:
Comp.1
S 2019
Comp.2
Comp.3
Comp.4


Standard deviation
0.009317648 0.003370953 0.001467303 0.0002519268
Proportion of Variance 0.864317887 0.113127014 0.021433880 0.0006318435
Cumulative Proportion 0.864317887 0.977444901 0.998878781 0.9995106243
Comp.5
Comp.6
Standard deviation
0.0002032373 8.860665e-05
Proportion of Variance 0.0004112140 7.816163e-05
Cumulative Proportion 0.9999218384 1.000000e+00


\end{verbatim} \end{framed}


The R-output shows that the proportion of variance explained by the first two principal
components is c.98\%, and by the first three components c.99%.

Therefore it would be reasonable to reduce the dimensions of the dataset by using the
first two (or three) principal components.

%%%%%%%%%%%%%%%%%%%%%%%%%%%%%%%%%%%%%%%%%%%%%%%%%%%%%
\newpage 
Candidates performed generally well in this question. Part (i) was very well answered,
with some partial answers in (i)(b) where many candidates observed a relationship in
individual years without drawing out the overall trend. Answers in part (ii) were also
satisfactory. Note that in part (ii) the princomp() function can alternatively be used in R.

\end{document}
